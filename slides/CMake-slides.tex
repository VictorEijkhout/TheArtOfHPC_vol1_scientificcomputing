%%%%%%%%%%%%%%%%%%%%%%%%%%%%%%%%%%%%%%%%%%%%%%%%%%%%%%%%%%%%%%%%
%%%%%%%%%%%%%%%%%%%%%%%%%%%%%%%%%%%%%%%%%%%%%%%%%%%%%%%%%%%%%%%%
%%%%
%%%% This text file is part of the source of slides for
%%%% `Introduction to High-Performance Scientific Computing'
%%%% by Victor Eijkhout, copyright 2012-2020
%%%%
%%%%%%%%%%%%%%%%%%%%%%%%%%%%%%%%%%%%%%%%%%%%%%%%%%%%%%%%%%%%%%%%
%%%%%%%%%%%%%%%%%%%%%%%%%%%%%%%%%%%%%%%%%%%%%%%%%%%%%%%%%%%%%%%%

\lstset{language=bash}

\begin{numberedframe}{Building software the old way}
  Using `GNU Autotools':
\begin{verbatim}
./configure
make
make install
\end{verbatim}
\end{numberedframe}

\begin{numberedframe}{User vs system packages}
  The \n{make install} often tries to copy
  to a system directory. If you're not the admin, do:
\begin{verbatim}
./configure --prefix=/home/yourname/mypackages
\end{verbatim}
with a location of your choice.
\end{numberedframe}

\begin{numberedframe}{Building with CMake}
  \begin{itemize}
  \item Either replace only the configure stage
\begin{lstlisting}
cmake ## argments
make
make install
\end{lstlisting}
or
  \item do everything with CMake:
\begin{lstlisting}
cmake ## argments
cmake --build ## stuff
cmake --install ## stuff
\end{lstlisting}
  \end{itemize}
(The second one is portable to non-Unix environments.)
\end{numberedframe}

\begin{numberedframe}{Directory structure}
  \hbox to \textwidth\bgroup
  \begin{minipage}{.45\textwidth}
    \tikzsetnextfilename{cmake-insource}
    \begin{tikzpicture}[dirtree]
      \node{dir}
      child { node {src} child { node {build} }
      }
      child { node {install} }
      ;
    \end{tikzpicture}
  \end{minipage}
  \hss
  \begin{minipage}{.45\textwidth}
    \tikzsetnextfilename{cmake-outsource}
    \begin{tikzpicture}[dirtree]
      \node{dir}
      child { node {src} }
      child { node {build} }
      child { node {install} }
      ;
    \end{tikzpicture}
  \end{minipage}
  \egroup
  \begin{itemize}
  \item In-source build: pretty common
  \item Out-of-source build: cleaner because never touches the source tree
  \end{itemize}
\begin{lstlisting}
\end{lstlisting}
\end{numberedframe}

\begin{numberedframe}{Out-of-source build}
  \begin{itemize}
  \item Work from a build directory
  \item Specify prefix and location of \n{CMakeLists.txt}
  \end{itemize}
\begin{lstlisting}
ls some_package_1.0.0 # we are outside the source
ls some_package_1.0.0/CMakeLists.txt # source contains cmake file
mkdir builddir && cd builddir # goto build location
cmake -D CMAKE_INSTALL_PREFIX=../installdir \
      ../some_package_1.0.0
make
make install
\end{lstlisting}
\end{numberedframe}

\lstset{language=CMake}

\begin{numberedframe}{The CMakeLists file}
  \begin{itemize}
  \item Which cmake version is needed for this file?\\
    (CMake has undergone quite some evolution!)
  \item Give a name to your project.
  \end{itemize}
\begin{lstlisting}
cmake_minimum_required( VERSION 3.12 )
project( myproject VERSION 1.0 )
\end{lstlisting}
\end{numberedframe}

\begin{numberedframe}{Target philosoph}
  \begin{itemize}
  \item Declare a target: something that needs to be built
  \item specify what is needed for it
  \end{itemize}
\begin{lstlisting}
add_executable( myprogram program.cxx )
install( TARGETS myprogram DESTINATION . )
\end{lstlisting}
Use of macros:
\begin{lstlisting}
add_executable( ${PROJECT_NAME} program.cxx )
\end{lstlisting}
\end{numberedframe}

\begin{numberedframe}{}
  \begin{itemize}
  \item 
  \end{itemize}
\begin{lstlisting}
\end{lstlisting}
\end{numberedframe}

\begin{numberedframe}{}
  \begin{itemize}
  \item 
  \end{itemize}
\begin{lstlisting}
\end{lstlisting}
\end{numberedframe}

