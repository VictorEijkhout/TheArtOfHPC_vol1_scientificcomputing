%%%%%%%%%%%%%%%%%%%%%%%%%%%%%%%%%%%%%%%%%%%%%%%%%%%%%%%%%%%%%%%%
%%%%%%%%%%%%%%%%%%%%%%%%%%%%%%%%%%%%%%%%%%%%%%%%%%%%%%%%%%%%%%%%
%%%%
%%%% This text file is part of the source of 
%%%% `The Art of HPC, vol 1: The Science of Computing'
%%%% by Victor Eijkhout, copyright 2012-2022
%%%%
%%%% This book is distributed under a Creative Commons Attribution 3.0
%%%% Unported (CC BY 3.0) license and made possible by funding from
%%%% The Saylor Foundation \url{http://www.saylor.org}.
%%%%
%%%% hpc_git.tex : git source control
%%%%
%%%%%%%%%%%%%%%%%%%%%%%%%%%%%%%%%%%%%%%%%%%%%%%%%%%%%%%%%%%%%%%%
%%%%%%%%%%%%%%%%%%%%%%%%%%%%%%%%%%%%%%%%%%%%%%%%%%%%%%%%%%%%%%%%

\documentclass[11pt,headernav]{beamer}

\usepackage{beamerthemeTACC}
\parskip=.5\baselineskip plus .5\baselineskip
\input semester

\setcounter{tocdepth}{1}
\AtBeginSection[]
{
  \begin{numberedframe}
    \frametitle{Table of Contents}
    \tableofcontents[currentsection]
  \end{numberedframe}
}

\usepackage{pslatex}
\usepackage{amsmath,comment,multirow,multicol} %% ,arydshln

\setbeamertemplate{footline}{Eijkhout: programming}

\newdimen\unitindent \unitindent=20pt

\usepackage{comment}

\input slidemacs
\input snippetmacs
\input scimacs

\begin{document}
\parskip=10pt plus 5pt minus 3pt

\title{Source Code Control through Git}
\author{Victor Eijkhout}
\date{2022}

\begin{frame}{}
  \titlepage
\end{frame}

\begin{numberedframe}{Justification}
  Source code control packages, such as Git,
  are essential for synchronizing software between
  developers, or multiple accounts for a single developer.
  They also allow you to keep a history of changes,
  and roll them back when needed.
\end{numberedframe}

\begin{numberedframe}{Preliminaries}
  Create an account on \n{github.com}

  Pick a good name, not referring to this class,
  so that you can keep it for a while.

  Authentication setup:
  \begin{itemize}
  \item Find the \n{id_rsa.pub} file (your `public key') in your \n{.ssh} directory;
    copy the contents.
  \item
    Go to personal settings, section `SSH keys' and add a key
    for the cluster. You may need one that is specific to the login node!
  \end{itemize}

\end{numberedframe}

\begin{numberedframe}{Creating a repository}
  \begin{enumerate}
  \item If you have a directory with material,
    you can declare it to be(come) a repository:\\
    \n{git init}
  \item Easier:
    \begin{enumerate}
    \item Make a new repository on \n{github.com}
    \item Do \n{git clone} with it.
    \end{enumerate}
  \end{enumerate}
  Github notes:
  \begin{itemize}
  \item On TACC machines, use \n{ssh} to clone a repo, not \n{https}
  \item See the point about ssh keys above.
  \end{itemize}
\end{numberedframe}

\begin{numberedframe}{Adding files}
  \begin{itemize}
  \item Create a file
  \item Do \n{git status}
  \item Do \n{git add yourfile}
  \item Enter message: \n{git commit -m "this is what I did"}
  \item do \n{git push}
  \item Check the \n{github.com} page for your repository.
  \end{itemize}  
\end{numberedframe}

\begin{numberedframe}{Changes to files}
  \begin{itemize}
  \item Edit the file that was added to the repo
  \item Explore \n{git status} and \n{git diff}
  \item Add and commit and push again.
  \end{itemize}
\end{numberedframe}

\begin{numberedframe}{Collaboration}
  \begin{itemize}
  \item Clone a repository from someone else
  \item (make sure you have permission to push to it)
  \item create a file and add/commit/push it
  \item The original owner can pull it.
  \end{itemize}
\end{numberedframe}

\begin{numberedframe}{Merging changes}
  \begin{itemize}
  \item Start with a file that is longer than a couple of lines
  \item Two people edit the same file, one at the top, the other at the bottom.
  \item Both add/commit/push
  \item Do you get an error message? Pull before push.
  \item Are both changes visible in the file?
  \end{itemize}
\end{numberedframe}

\begin{numberedframe}{Merging conflicting changes}
  \begin{itemize}
  \item Make changes on two adjacent lines
  \item Merging should fail
  \item Do a manual edit to resolve the conflict
  \item (Did you get some full-screen tool?)
  \end{itemize}
\end{numberedframe}

\begin{numberedframe}{Branches}
  Branches are good for experiments
  \begin{itemize}
  \item Create a branch\\ \n{git branch dev}\\ \n{git checkout dev}
  \item which branches do you have?\\ \n{git branch -a}\\
    which one are you currently on?
  \end{itemize}
\end{numberedframe}

\begin{numberedframe}{working with branches}
  \begin{itemize}
  \item While on the \n{dev} branch, make an edit to a file
  \item Check that the file is not edited no the main branch
  \item Go to the main branch, make a non-conflicting change 
  \end{itemize}
\end{numberedframe}

\begin{numberedframe}{Merging}
  \begin{itemize}
  \item See the difference between branches\\
    \n{git diff main dev}
  \item Merge while on the main branch:\\
    \n{git merge dev}
  \item Inspect.
  \end{itemize}
\end{numberedframe}

\end{document}

\begin{numberedframe}{}
  \label{sl-tut:}
  \begin{itemize}
  \item 
  \end{itemize}
\end{numberedframe}

