%%%%%%%%%%%%%%%%%%%%%%%%%%%%%%%%%%%%%%%%%%%%%%%%%%%%%%%%%%%%%%%%
%%%%%%%%%%%%%%%%%%%%%%%%%%%%%%%%%%%%%%%%%%%%%%%%%%%%%%%%%%%%%%%%
%%%%
%%%% This text file is part of the source of 
%%%% `The Art of HPC, vol 1: The Science of Computing'
%%%% by Victor Eijkhout, copyright 2012-2022
%%%%
%%%% This book is distributed under a Creative Commons Attribution 3.0
%%%% Unported (CC BY 3.0) license and made possible by funding from
%%%% The Saylor Foundation \url{http://www.saylor.org}.
%%%%
%%%% hpc_parallel.tex : basic theory of parallelism
%%%%
%%%%%%%%%%%%%%%%%%%%%%%%%%%%%%%%%%%%%%%%%%%%%%%%%%%%%%%%%%%%%%%%
%%%%%%%%%%%%%%%%%%%%%%%%%%%%%%%%%%%%%%%%%%%%%%%%%%%%%%%%%%%%%%%%

\documentclass[11pt]{beamer}

\usepackage{beamerthemeTACC}
\parskip=.5\baselineskip plus .5\baselineskip
\input semester
%\beamertemplatenavigationsymbolsempty

\usepackage{pslatex}
\usepackage{amsmath,comment,multirow,multicol} %% ,arydshln

%\setbeamertemplate{footline}{Eijkhout: HPC intro}

\newdimen\unitindent \unitindent=20pt

\input slidemacs
\input snippetmacs
\input scimacs

\excludecomment{answer}

\begin{document}
\parskip=10pt plus 5pt minus 3pt

\title{What is parallelism?}
\author{\hpcteachers}
\date{\hpcsemester}

\begin{frame}
  \titlepage
\end{frame}

\begin{frame}{Justification}
  Parallel computing has been a necessity for decades in computational science.
  Here we discuss some of the basic concepts.
  Actual parallel programming will be discussed in other lectures.
\end{frame}

\input Parallelism-slides

\end{document}

\begin{frame}[containsverbatim]{}
  \begin{itemize}
  \item 
  \end{itemize}
\begin{lstlisting}
\end{lstlisting}
\end{frame}

