%%%%%%%%%%%%%%%%%%%%%%%%%%%%%%%%%%%%%%%%%%%%%%%%%%%%%%%%%%%%%%%%
%%%%%%%%%%%%%%%%%%%%%%%%%%%%%%%%%%%%%%%%%%%%%%%%%%%%%%%%%%%%%%%%
%%%%
%%%% This text file is part of the source of 
%%%% `The Art of HPC, vol 4: HPC Carpentry
%%%% by Victor Eijkhout, copyright 2012-2023
%%%%
%%%% This book is distributed under a Creative Commons Attribution 3.0
%%%% Unported (CC BY 3.0) license and made possible by funding from
%%%% The Saylor Foundation \url{http://www.saylor.org}.
%%%%
%%%% hpc_cmake.tex : CMake tutorial
%%%%
%%%%%%%%%%%%%%%%%%%%%%%%%%%%%%%%%%%%%%%%%%%%%%%%%%%%%%%%%%%%%%%%
%%%%%%%%%%%%%%%%%%%%%%%%%%%%%%%%%%%%%%%%%%%%%%%%%%%%%%%%%%%%%%%%

\documentclass[10pt]{beamer}

\input courseformat
\input tikzplot
\def\qrcode{qrvol1}

\begin{document}
\input semester
\title[CMake tutorial]{Building projects with CMake}
\author[Eijkhout]{Victor Eijkhout}
\date{\hpcsemester}

\maketitle

\begin{frame}{Justification}
  CMake is a portable build system that is
  becoming a \emph{de facto} standard for C++ package management.

  Also usable with C and Fortran.
\end{frame}

\begin{frame}{Table of contents}
  \tableofcontents
\end{frame}

%% part 1: use cmake-based library
\sectionframe{Help! This software uses CMake!}
\input CMake-slides
\input CMakepc-slides
\input CMakefind-slides

%% part 2: make your own cmake setup
\sectionframe{Help! I want to write CMake myself!}
\input CMakedesign-slides
\input CMakeother-slides

%% part 3: miscellaneous
\sectionframe{Help! I want people to use my CMake package!}
\input CMakepcin-slides

\sectionframe{Example libraries}
\input CMakempiomp-slides.tex
\input CMakedata-slides.tex
\input CMakeexamples-slides

\end{document}
