% -*- latex -*-
%%%%%%%%%%%%%%%%%%%%%%%%%%%%%%%%%%%%%%%%%%%%%%%%%%%%%%%%%%%%%%%%
%%%%%%%%%%%%%%%%%%%%%%%%%%%%%%%%%%%%%%%%%%%%%%%%%%%%%%%%%%%%%%%%
%%%%
%%%% This text file is part of the source of slides for
%%%% `The Art of HPC, vol 1: The Science of Computing'
%%%% by Victor Eijkhout, copyright 2012-2024
%%%%
%%%% CMake-slides.tex : cmake slides
%%%% see hpc_cmake.tex for driver file
%%%%
%%%%%%%%%%%%%%%%%%%%%%%%%%%%%%%%%%%%%%%%%%%%%%%%%%%%%%%%%%%%%%%%
%%%%%%%%%%%%%%%%%%%%%%%%%%%%%%%%%%%%%%%%%%%%%%%%%%%%%%%%%%%%%%%%

\lstset{language=bash}

\subsectionframe{Using CMake packages through pkgconfig}

\begin{numberedframe}{What are we talking here?}
  You have just installed a CMake-based library.

  Now you need it in your own code, or in another library.

  How easy can we make that?
\end{numberedframe}

\begin{numberedframe}{Problem}
  You want to install an application/package\\
  \dots~which needs 2 or 3 other packages.
\lstset{language=Bash}
\begin{lstlisting}
gcc -o myprogram myprogram.c \
    -I/users/my/package1/include \ 
    -L/users/my/package1/lib \
    -I/users/my/package2/include/package \ 
    -L/users/my/package2/lib64
\end{lstlisting}
or:
\begin{lstlisting}
cmake \
    -D PACKAGE1_INC=/users/my/package1/include \ 
    -D PACKAGE1_LIB=/users/my/package1/lib \ 
    -D PACKAGE2_INC=/users/my/package2/include/package \ 
    -D PACKAGE2_LIB=/users/my/package2/lib64 \ 
    ../newpackage
\end{lstlisting}
Can this be made simpler?
\end{numberedframe}

\begin{numberedframe}{Finding packages with `pkg config'}
  \begin{itemize}
  \item Many packages come with a \n{package.pc} file
  \item Add that location to \n{PKG_CONFIG_PATH}
  \item The package can now be found by other CMake-based packages.
  \end{itemize}
\end{numberedframe}

\begin{numberedframe}{Package config settings}
 Let's say you've installed a library with CMake.
  
Somewhere in the installation is a \n{.pc} file:
\begin{lstlisting}[language=bash]
find $TACC_SMTHNG_DIR -name \*.pc
${TACC_SMTHNG_DIR}/share/pkgconfig/smthng3.pc    
\end{lstlisting}
That location needs to be on the \n{PKG_CONFIG_PATH}:
\begin{lstlisting}[language=bash]
export PKG_CONFIG_PATH=${TACC_SMTHNG_DIR}/share/pkgconfig:${PKG_CONFIG_PATH}
\end{lstlisting}
\end{numberedframe}

\begin{numberedframe}{Example: eigen}
  Can you find the \lstinline{.pc} file
  in the Eigen installation?
\end{numberedframe}

\begin{numberedframe}{Scenario 1: finding without cmake}

  Packages with a \n{.pc} file can be found
  through the \indexunix{pkg-config} command:
  
\begin{lstlisting}
gcc -o myprogram myprogram.c \
    $( pkg-config --cflags package1 ) \ 
    $( pkg-config --libs package1 )
\end{lstlisting}

In a makefile:
\begin{lstlisting}
CFLAGS = -g -O2 $$( pkg-config --cflags package1 )
\end{lstlisting}
\end{numberedframe}

\begin{numberedframe}{Example: eigen}
\begin{lstlisting}
#include "Eigen/Core"
int main(int argc,char **argv) {
  return 0;
}
\end{lstlisting}
Can you compile this on the commandline, using \lstinline{pkg-config}?
Small problem: `eigen' wants to be called `eigen3'.
\end{numberedframe}

\begin{numberedframe}{Scenario 2: finding from CMake}
  You are installing a CMake-based library\\
  and it needs Eigen, which is also CMake-based
  \begin{enumerate}
  \item you install Eigen with CMake, as above
  \item you add the location of \n{eigen.pc} to
    \n{PKG_CONFIG_PATH}
  \item you run the installation of the higher library:\\
    this works because it can now find Eigen.
  \end{enumerate}
\end{numberedframe}

\begin{numberedframe}{Lifting the veil}
  So how does a CMake install find libraries such as Eigen?

  \lstset{numbers=left,numberstyle=\tiny}
  \lstinputlisting{tutorials/cmake/eigen/CMakeLists.txt}
  Note 1: header-only so no library, otherwise
  \n{PACKAGE_LIBRARY_DIRS} and \n{PACKAGE_LIBRARIES} defined.

  Note 2: you will learn how to write these configurations
  in the second part.
\end{numberedframe}

\begin{numberedframe}{Summary for now}
  \begin{itemize}
  \item You can use CMake to install libraries;
  \item You can use these libraries from commandline~/ makefile;
  \item You can let other CMake-based libraries find them.
  \end{itemize}
\end{numberedframe}

\endinput

\begin{numberedframe}{OpenMP from Fortran}
\begin{lstlisting}
\end{lstlisting}
\end{numberedframe}

\begin{numberedframe}{OpenMP from Fortran}
\begin{lstlisting}
\end{lstlisting}
\end{numberedframe}

