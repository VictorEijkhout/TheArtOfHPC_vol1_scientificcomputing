%%%%%%%%%%%%%%%%%%%%%%%%%%%%%%%%%%%%%%%%%%%%%%%%%%%%%%%%%%%%%%%%
%%%%%%%%%%%%%%%%%%%%%%%%%%%%%%%%%%%%%%%%%%%%%%%%%%%%%%%%%%%%%%%%
%%%%
%%%% This text file is part of the source of slides for
%%%% `The Art of HPC, vol 1: The Science of Computing'
%%%% by Victor Eijkhout, copyright 2012-2022
%%%%
%%%%%%%%%%%%%%%%%%%%%%%%%%%%%%%%%%%%%%%%%%%%%%%%%%%%%%%%%%%%%%%%
%%%%%%%%%%%%%%%%%%%%%%%%%%%%%%%%%%%%%%%%%%%%%%%%%%%%%%%%%%%%%%%%

\begin{numberedframe}{Analysis basics}
  \begin{itemize}
  \item Measurements: repeated and controlled\\
    beware of transients, do you know where your data is?
  \item Document everything
  \item Script everything
  \end{itemize}
\end{numberedframe}

\begin{numberedframe}{Compiler options}
  \begin{itemize}
  \item Defaults are a starting point
  \item use reporting options: \texttt{-opt-report}, \texttt{-vec-report}\\
    useful to check if optimization happened~/ could not happen
  \item test numerical correctness before/after optimization change (there are options for numerical corretness)
  \end{itemize}
\end{numberedframe}

\begin{numberedframe}{Optimization basics}
  \begin{itemize}
  \item Use libraries when possible: don't reinvent the wheel
  \item Premature optimization is the root of all evil (Knuth)
  \end{itemize}
\end{numberedframe}

\begin{numberedframe}{Code design for performance}
  \begin{itemize}
  \item Keep inner loops simple: no conditionals, function calls, casts
  \item Avoid small functions: try macros or inlining
  \item Keep in mind all the cache,TLB, SIMD stuff from before
  \item SIMD: Fortran array syntax helps
  \end{itemize}
\end{numberedframe}

\begin{numberedframe}{Multicore / multithread}
  \begin{itemize}
  \item Use \texttt{numactl}: prevent process migration
  \item `first touch' policy: allocate data where it will be used
  \item Scaling behaviour mostly influenced by bandwidth
  \end{itemize}
\end{numberedframe}

\begin{numberedframe}{Multinode performance}
  \begin{itemize}
  \item Influenced by load balancing
  \item Use HPCtoolkit, Scalasca, TAU for plotting
  \item Explore `eager' limit (mvapich2: environment variables)
  \end{itemize}
\end{numberedframe}

\endinput

\begin{numberedframe}{}
  \begin{itemize}
  \item 
  \end{itemize}
\end{numberedframe}

\begin{numberedframe}{}
  \begin{itemize}
  \item 
  \end{itemize}
\end{numberedframe}

