%%%%%%%%%%%%%%%%%%%%%%%%%%%%%%%%%%%%%%%%%%%%%%%%%%%%%%%%%%%%%%%%
%%%%%%%%%%%%%%%%%%%%%%%%%%%%%%%%%%%%%%%%%%%%%%%%%%%%%%%%%%%%%%%%
%%%%
%%%% This text file is part of the source of 
%%%% `The Art of HPC, vol 1: The Science of Computing'
%%%% by Victor Eijkhout, copyright 2012-2022
%%%%
%%%% This book is distributed under a Creative Commons Attribution 3.0
%%%% Unported (CC BY 3.0) license and made possible by funding from
%%%% The Saylor Foundation \url{http://www.saylor.org}.
%%%%
%%%% hpc_course.tex : main file for the big HPC slide deck
%%%%
%%%%%%%%%%%%%%%%%%%%%%%%%%%%%%%%%%%%%%%%%%%%%%%%%%%%%%%%%%%%%%%%
%%%%%%%%%%%%%%%%%%%%%%%%%%%%%%%%%%%%%%%%%%%%%%%%%%%%%%%%%%%%%%%%

\documentclass[11pt,headernav]{beamer}

\usepackage{beamerthemeTACC}
\parskip=.5\baselineskip plus .5\baselineskip
\input semester

\setcounter{tocdepth}{1}
\AtBeginSection[]
{
  \begin{frame}
    \frametitle{Table of Contents}
    \tableofcontents[currentsection]
  \end{frame}
}

\usepackage{pslatex}
\usepackage{amsmath,comment,multirow,multicol} %% ,arydshln

\setbeamertemplate{footline}{Eijkhout: HPC intro}

\newdimen\unitindent \unitindent=20pt

\input slidemacs
\input snippetmacs
\input scimacs

\excludecomment{answer}
\excludecomment{simplified}
\includecomment{lotsamath}
\includecomment{diagonalstorage}

\begin{document}
\parskip=10pt plus 5pt minus 3pt

\title{Short course on High Performance Computing}
\author{Victor Eijkhout}
\date{2018}

\begin{frame}
  \titlepage
\end{frame}

\begin{frame}
  \tableofcontents
\end{frame}

\begin{frame}{Justification}

High Performance Computing is a field that brings together
algorithms, software, and hardware. This course
conveys the basics of computer architecture,
scientific algorithms, and how to code
algorithms to make them efficient on current hardware.

\end{frame}


\Level 0 {Processor Architecture}
\input ProcessorArchitecture-slides

\Level 0 {Parallelism}

\begin{frame}{Justification}
  Parallelism can be approached in several different ways.
  This session will discuss data paralellism versus instruction parallelism,
  issues in shared memory parallelism, parallel programming systems,
  the interconnects of distributed memory paralellism,
  scaling measures.
\end{frame}

%% \input life-slides
\input Parallelism-slides

\Level 0 {Computer arithmetic}

\begin{frame}{Justification}
  This short session will explain the basics of floating point
  arithmetic, mostly focusing on round-off and its influence
  on computations.
\end{frame}

\input ComputerArithmetic-slides

\Level 0 {Partial Differential Equations}

\begin{frame}{Justification}
  Partial differential equations are an important source of 
  large-scale engineering problems. Here we take a look at
  their computational aspects.
\end{frame}

\input PartialDifferentialEquation-slides.tex

\Level 0 {Linear algebra}

\begin{frame}{Justification}
  Many algorithms are based in linear algebra, 
  including some non-obvious ones such as graph algorithms.
  This session will mostly discuss aspects of 
  solving linear systems, focusing on those
  that have computational ramifications.
\end{frame}

\newenvironment{beamdisplayeq}%
 {\begin{equation}\small}{\end{equation}}

\Level 1 {Essential aspects of LU factorization}

\input LinearAlgebra-slides

\Level 1 {Sparse matrices: storage and algorithms}

\input SparseMatrices-slides

%% \Level 1 {Direct approaches to sparse matrix factorization}

%% \input DomainDecomp-slides

\Level 1 {Iterative methods, basic concepts and available methods}

\input IterativeMethods-slides

\Level 0 {High performance linear algebra}

\begin{frame}{Justification}
  Bringing architecture-awareness to linear algebra,
  we discuss how high performance results from
  using the right formulation and implementation of algorithms.
\end{frame}

\input Collectives
\input DenseMVP
\input SparseMVP
\input LatencyHiding
\input IterativeComputational
\input ParallelLU
\input IncompleteLU
\input ImplicitParallel
\input MulticoreBlock

\Level 0 {Applications}

\begin{frame}{Justification}
  We briefly discuss two applications that,
  while at first glance not linear-algebra like, surprisingly 
  can be covered by the foregoing concepts.
\end{frame}

\Level 1 {N-body problems: naive and equivalent formulations}
%\input BH-slides
\input Nbody-slides

\Level 1 {Graph analytics, interpretation as sparse matrix problems}

\input GraphAlgorithms-slides

\Level 0 {Parallel programming topics}

\begingroup \catcode`\_=12
\let\indexmpishow\texttt
\input MPI-slides
\endgroup

\Level 1 {Profiling and debugging; optimization and programming strategies.}

\input ProfilingOptimization-slides
\input Debug-slides

%% \begin{frame}
%%   \includegraphics[scale=.08]{bookcover}
%% \end{frame}

\newenvironment{theindex}{\begin{itemize}}{\end{itemize}}
\let\indexspace\par
\def\subitem{\par\indent}

%% \begin{frame}{Index}
%% \small
%% \begin{multicols}{2}
%% \printindex  
%% \end{multicols}
%% \end{frame}

\end{document}
