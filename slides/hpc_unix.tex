%%%%%%%%%%%%%%%%%%%%%%%%%%%%%%%%%%%%%%%%%%%%%%%%%%%%%%%%%%%%%%%%
%%%%%%%%%%%%%%%%%%%%%%%%%%%%%%%%%%%%%%%%%%%%%%%%%%%%%%%%%%%%%%%%
%%%%
%%%% This text file is part of the source of 
%%%% `Introduction to High-Performance Scientific Computing'
%%%% by Victor Eijkhout, copyright 2012-2022
%%%%
%%%% This book is distributed under a Creative Commons Attribution 3.0
%%%% Unported (CC BY 3.0) license and made possible by funding from
%%%% The Saylor Foundation \url{http://www.saylor.org}.
%%%%
%%%% hpc_unix.tex : unix tutorial
%%%%
%%%%%%%%%%%%%%%%%%%%%%%%%%%%%%%%%%%%%%%%%%%%%%%%%%%%%%%%%%%%%%%%
%%%%%%%%%%%%%%%%%%%%%%%%%%%%%%%%%%%%%%%%%%%%%%%%%%%%%%%%%%%%%%%%

\documentclass[11pt,headernav]{beamer}

\usepackage{beamerthemeTACC}
\parskip=.5\baselineskip plus .5\baselineskip
\input semester

\setcounter{tocdepth}{1}
%% \AtBeginSection[]
%% {
%%   \begin{numberedframe}
%%     \frametitle{Table of Contents}
%%     %\tableofcontents[currentsection]
%%   \end{numberedframe}
%% }

\usepackage{pslatex}
\usepackage{amsmath,multicol}
% multirow,
% ,arydshln

\setbeamertemplate{footline}{Eijkhout: shell\hfill \arabic{page}}

\newdimen\unitindent \unitindent=20pt

\usepackage{comment}

\input slidemacs

\input scimacs

%\usepackage{makeidx}
%\makeindex

\begin{document}
\parskip=10pt plus 5pt minus 3pt

\title{Introduction to Unix}
\author{Victor Eijkhout}
\date{2021}

\begin{frame}
  \titlepage
\end{frame}

\tocslide

\begin{numberedframe}{Justification}
  Unix, in particular Linux, is the \textsl{de facto}
  operating system in \ac{HPC}.
\end{numberedframe}

\Level 0 {Files and such}

\begin{numberedframe}{\texttt{ls}, \texttt{touch}}
  \label{sl-lnx:ls}
  \begin{itemize}
  \item List files: \n{ls}
  \item Maybe your account is still empty: do \n{touch newfile},
    then \n{ls} again.
  \item Options: \n{ls -l} or for specific file \n{ls -l newfile}.
  \end{itemize}
\end{numberedframe}

%% \begin{numberedframe}{}
%%   \includegraphics[scale=.4]{unixshots/1 ls empty}
%% \end{numberedframe}

%% \begin{numberedframe}{}
%%   \includegraphics[scale=.4]{unixshots/2 ls newfile}
%% \end{numberedframe}

\begin{numberedframe}{Display / add to file: \texttt{cat}}
  \label{sl-lnx:cat}
  \begin{itemize}
  \item Display a file: \n{cat myfile}
  \item Put something in a file: \n{cat > myfile}\\
    end with Control-D.\\
    Or use an editor, but this is sometimes still useful.
  \item Now \n{cat} it again.
  \item Do \n{cat >> myfile} and enter some text.
    What did this do?
  \end{itemize}
\end{numberedframe}

%% \begin{numberedframe}{}
%%   \includegraphics[scale=.4]{unixshots/3 cat into myfile}
%% \end{numberedframe}

%% \begin{numberedframe}{}
%%   \includegraphics[scale=.4]{unixshots/4 cat the myfile}
%% \end{numberedframe}

\begin{numberedframe}{\texttt{cp}, \texttt{mv}, \texttt{rm}}
  \label{sl-lnx:cpmv}
  \begin{itemize}
  \item Copy: \n{cp file1 file2}\\
    Do this, check that it's indeed a copy.
  \item Rename or `move': \n{mv file1 file2}\\
    check that the original file doesn't exist any more.
  \item Remove: \n{rm myfile}\\
    This is irrevocable!
  \end{itemize}
\end{numberedframe}

%% \begin{numberedframe}{}
%%   \includegraphics[scale=.4]{unixshots/5 copy myfile}
%% \end{numberedframe}

%% \begin{numberedframe}{}
%%   \includegraphics[scale=.4]{unixshots/6 mv new file to old}
%% \end{numberedframe}

%% \begin{numberedframe}{}
%%   \includegraphics[scale=.4]{unixshots/7 rm old}
%% \end{numberedframe}

\begin{numberedframe}{Dealing with large (text) files}
  \label{sl-lnx:headtail}
  \begin{itemize}
  \item If a file is larger than your screen:\\
    \n{less yourfile}
  \item If the start or end is interesting enough:\\
    \n{head yourfile}, \n{tail yourfile}
  \item Explore options: \n{head -n 5 yourfile}
  \end{itemize}
\end{numberedframe}

%% \begin{numberedframe}{}
%%   \includegraphics[scale=.4]{unixshots/8 less lorem}
%% \end{numberedframe}

%% \begin{numberedframe}{}
%%   \includegraphics[scale=.4]{unixshots/9 head lorem}
%% \end{numberedframe}

\begin{exercise}{Put the pieces together}
  How would you display the 3rd line of a file?
\end{exercise}

\Level 0 {Directories}

\begin{numberedframe}{Directories}
  \label{sl-lnx:cd}
  \begin{itemize}
  \item Make a subdirectory `folder': \n{mkdir newdir}
  \item Check where you are: \n{pwd}
  \item Now go to the new directory: \n{cd newdir} and \n{pwd}\\
    `change directory' and `present working directory'
  \item Back to your home directory: \n{cd} without further arguments.
  \end{itemize}
\end{numberedframe}

%% \begin{numberedframe}{}
%%   \includegraphics[scale=.4]{unixshots/10 newdir}
%% \end{numberedframe}

\begin{numberedframe}{Paths}
  \label{sl-lnx:path}
  \begin{itemize}
  \item Do:
    \begin{enumerate}
    \item \n{cd newdir} \item \n{touch nested_file} \item \n{cd}
    \end{enumerate}
  \item Now: \n{ls newdir/nested_file}
  \item That is called a path
    \begin{itemize}
    \item Relative path: does not start with slash
    \item Absolute path (such as \n{pwd} output): starts at root
    \end{itemize}
  \end{itemize}
\end{numberedframe}

%% \begin{numberedframe}{}
%%   \includegraphics[scale=.4]{unixshots/11 nested file}
%% \end{numberedframe}

\begin{numberedframe}{More paths}
  \label{sl-lnx:morepath}
  \begin{itemize}
  \item Path to your home directory: tilde
    \n{cd ~}
  \item Current directory:~\n{.}
  \item Going out of a directory:~\n{cd ..}\\
    (confusing: do you call this a level up or down?)
  \item You can use this in paths: \n{ls newdir/subdir1/../subdir2}
  \end{itemize}
  %% Exercise: copy the lorem ipsum file from the repo to a new directory.
\end{numberedframe}

%% \begin{numberedframe}{}
%%   \includegraphics[scale=.4]{unixshots/12 more dir file}
%% \end{numberedframe}

%% \begin{numberedframe}{}
%%   \includegraphics[scale=.4]{unixshots/13 dot path}
%% \end{numberedframe}

\begin{exercise}{Paths}
  After the following commands:
\begin{verbatim}
mkdir somedir
touch somedir/somefile
\end{verbatim}
Give at least two ways of specifying the path to \n{somefile}
from the current directory
for instance for the \n{ls} command.\\
Same after doing \n{cd somedir}
\end{exercise}

\Level 0 {Redirection, pipes}

\begin{numberedframe}{In/Output redirection}
  \label{sl-lnx:redirect}
  \begin{itemize}
  \item There are three standard files: \n{stdin}, \n{stdout}, \n{stderr}
  \item Normally connected to keyboard, screen, and screen respectively.
  \item Redirection: standard out to file:\\
    \n{ls > directorycontents}\\
    (actually, screen is a file, so it is really a \textbf{re}direct)
  \item Standard in from file:
    \n{mail < myfile}\\
    (actually, the keyboard is also a file, so again \textbf{re}direction)
  \end{itemize}
\end{numberedframe}

\begin{exercise}{}
  Make a copy of a file, using redirection (so no \n{cp} command).
\end{exercise}

\begin{numberedframe}{Advanced: splitting out and err}
  \label{sl-lnx:stderr}
  \begin{itemize}
  \item Sometimes you want to split standard out and error:
  \item Use \n{stdout}$=1$ and \n{stderr}$=2$:\\
    \n{myprogram 1>results.out 2>results.err}
  \item Very useful: get rid of errors:\\
    \n{myprogram 2>/dev/null}
  \end{itemize}
\end{numberedframe}

\begin{numberedframe}{Pipes}
  \label{sl-lnx:pipe}
  \begin{itemize}
  \item Redirection is command-to-file.
  \item Pipe: command-to-command\\
    \n{ls | wc -l}\\
    (what does this do?)
  \item Unix philosophy: small building blocks, put together.
  \end{itemize}
\end{numberedframe}

\begin{numberedframe}{More command sequencing}
  \label{sl-lnx:backtick}
  More complicated case of one command providing input for another:
\begin{verbatim}
echo The line count is wc -l foo
\end{verbatim}
where \n{foo} is the name of an existing file.

Use backquotes or command macro:
\begin{verbatim}
echo The line count is `wc -l foo`
echo "There are $( wc -l foo ) lines"
\end{verbatim}
\end{numberedframe}

\begin{exercise}{}
  This way \n{wc} prints the file name.
  Can you figure out a way to prevent that from happening?
\end{exercise}

\Level 0 {Permissions}

\begin{numberedframe}{Basic permissions}
  \label{sl-lnx:rwx}
  \begin{itemize}
  \item Three degrees of access: user/group/other
  \item three types of access: read/write/execute
  \end{itemize}
  \[ \begin{array}{ccc}
    \hline user&group&other\\ \hline rwx&rwx&rwx \\ \hline
  \end{array}
  \]
  Example: \n{rw-r-----}:\\
  owner read-write, group read, world nothing
\end{numberedframe}

\begin{numberedframe}{Permission setting}
  \label{sl-lnx:755}
  \begin{itemize}
  \item Add permissions \n{chmod g+w myfile}
  \item recursively: \n{chmod -R o-r mydirectory}
  \item Permissions are an octal number:
    \n{chmod 644 myfile}
  \end{itemize}
\end{numberedframe}

%% \begin{numberedframe}{}
%%   \includegraphics[scale=.4]{unixshots/14 mod plus x}
%% \end{numberedframe}

\begin{numberedframe}{Share files}
  \label{sl-lnx:share}
  \begin{itemize}
  \item Make a file in your \n{$WORK} file system,
    and make it visible to the world.
  \item Ask a fellow student to view it.
  \item $\Rightarrow$ also necessary to make \n{$WORK} readable.\\
    (Not a good idea to make \n{$HOME} readable.)
  \end{itemize}
\end{numberedframe}

\begin{numberedframe}{The \texttt{x} bit}
  \label{sl-lnx:x}
  The \n{x} bit has two meanings:
  \begin{itemize}
  \item For regular files: executable.
  \item For directories: you can go into them.
  \item Make all directories viewable:\\
    \n{chmod -R g+X,o+X rootdir}
  \end{itemize}
\end{numberedframe}

\Level 0 {Shell programming}

\begin{numberedframe}{Command execution}
  \label{sl-lnx:execute}
  \begin{itemize}
  \item Some shell commands are built-in, however most are programs.
  \item \n{which ls}
  \item Exercise: what can you find out about the \n{ls} program?
  \item Programs can be called directly: \n{/bin/ls}
    or found along the search path \n{$PATH}:\\
    \n{echo $PATH}
  \end{itemize}
\end{numberedframe}

\begin{numberedframe}{The PATH variable}
  \begin{itemize}
  \item The \n{PATH} variable is set by the system
  \item You can add in the \n{.bashrc} file
  \item TACC module system~\ldots
  \item Temporary:
\begin{verbatim}
export PATH=/my/bin/dir:${PATH}
\end{verbatim}
\item Changes to \n{.bashrc} take effect next time you log in\\
  or \n{source .bashrc} for immediate results.
  \end{itemize}
\end{numberedframe}

\begin{numberedframe}{Things that look like commands}
  \label{sl-lnx:alias}
  \begin{itemize}
  \item Use \n{alias} to give a new name to a command:\\
    \n{alias ls='ls -F'}\\
    \n{alias rm='rm -i'}
  \item There is a shell level \n{function} mechanism,
    not explained here.
  \end{itemize}
\end{numberedframe}

\begin{numberedframe}{Processes}
  \label{sl-lnx:ps}
\begin{tabular}{ll}
  \toprule
  \n{ps}&list (all) processes\\
  \n{kill}&kill a process\\
  \verb+CTRL-c+&kill the foreground job\\
  \verb+CTRL-z+&suspect the foreground job\\
  \n{jobs}&give the status of all jobs\\
  \n{fg}&bring the last suspended job to the foreground\\
  \verb+fg %3+&bring a specific job to the foreground\\
  \n{bg}&run the last suspended job in the background\\
  \bottomrule
\end{tabular}

Exercise: how many programs do you have running?
\end{numberedframe}

\begin{numberedframe}{Variables}
  \label{sl-lnx:var}
  \begin{itemize}
  \item \n{PATH} is a variable, built-in to the shell
    \item you can make your own variables:
\begin{verbatim}
a=5
echo $a
\end{verbatim}
No spaces around the equals!
  \end{itemize}
Exercise: what happens when you try to add two variables together?
\begin{verbatim}
a=3
b=5
\end{verbatim}
\end{numberedframe}

\begin{numberedframe}{Variable manipulation}
  \begin{itemize}
  \item Often you want to strip prefixes or suffixes from a variable:\\
    \n{program.c} $\Rightarrow$ \n{program}\\
    \n{/usr/bin/program} $\Rightarrow$ \n{program}
  \item Parameter expansion:
\begin{verbatim}
a=program.c
echo ${a%%.c}
a=/foo/bar/program.c
eecho ${a##*/}
\end{verbatim}
  \end{itemize}
\end{numberedframe}

\begin{numberedframe}{Conditionals}
  \label{sl-lnx:if}
\begin{itemize}
  \item Mostly text-based tests:
\begin{verbatim}
if [ $a = "foo" ] ; then
  echo "that was foo"
else
  echo "that was $a"
fi
\end{verbatim}
\item Single line:
{  \footnotesize
\begin{verbatim}
if [ $a = "foo" ] ; then echo "foo" ; else echo "something" ; fi
\end{verbatim}
}
Note the semicolons!\\
also spaces around square brackets.
  \end{itemize}
\end{numberedframe}

\begin{numberedframe}{Other conditionals}
  \begin{itemize}
\item Numerical tests:/
\begin{verbatim}
if [ $a -gt 2 ] ....
\end{verbatim}
\item File and directory:
{ \footnotesize
\begin{verbatim}
if [ -f $HOME ] ; then echo "exists" ; else echo "no such" ; fi
if [ -d $HOME ] ; then echo "directory!" ; else echo "file" ; fi
\end{verbatim}
}
  \end{itemize}
\end{numberedframe}

\begin{numberedframe}{Looping}
  \label{sl-lnx:for}
  \begin{itemize}
  \item Loop: for item in list\\
    the item is available as macro
\begin{verbatim}
for letter in a b c ; do echo $letter ; done
\end{verbatim}
\item Loop over files:
\begin{verbatim}
for file in * ; do echo $file ; done
\end{verbatim}
  \end{itemize}
  Exercises:
  \begin{enumerate}
  \item for each file, print its name and how many lines there are in it.
  \item loop through your files, print which ones are directories.
  \item for each C program, remove the object file.
  \end{enumerate}
\end{numberedframe}

\begin{numberedframe}{Numerical looping}

  \begin{itemize}
  \item Type \n{seq 1 5}
  \item Exercise: can you figure out how to loop $1\ldots5$?
\begin{verbatim}
n=12
## input
for i in ....... ; do echo $i ; done
## output
1
....
12
\end{verbatim}
  \end{itemize}
\end{numberedframe}

\Level 0 {Scripting}

\begin{numberedframe}{Script execution}
  \label{sl-lnx:script}
  \begin{itemize}
  \item Create a script \n{script.sh}:
\begin{verbatim}
#!/bin/bash
echo foo
\end{verbatim}
\item Can you execute this? Does the error suggest a remedy?
\item What is the remaining problem?
  \end{itemize}
\end{numberedframe}

\begin{numberedframe}{Arguments}

  \begin{itemize}
  \item You want to call \n{./script.sh myfile}
    \item Parameters are \n{$1} et cetera:
\begin{verbatim}
#!/bin/bash
echo "$1 is a file"
\end{verbatim}
\item How many arguments: \n{$#}
  \end{itemize}
\end{numberedframe}

\begin{numberedframe}{Exercise}

  Write a script that takes as input a file name argument, and reports how many
  lines are in that file.

  Edit your script to test whether the file has less than 10 lines
  (use the \n{foo -lt bar} test), and if it does, \n{cat} the
  file. 

  Add a
  test to your script so that it will give a helpful message if you call
  it without any arguments.

\end{numberedframe}

\begin{numberedframe}{Exercise}
  \label{sl-lnx:plag}
  Write a `plagiarism detector'.
  \begin{itemize}
  \item Write a script that accepts two argument: one text file and one directory\\
    \n{./yourscript.sh myfile targetdir}\\
    (the \n{.sh} extension is required for this exercise)
  \item Your script should compare the text file to the contents of the directory:
    \begin{itemize}
    \item If the file is different from anything in the directory, it
      should be copied into the directory;
      the script should not produce any output in this case.
    \item If the file is the same as a file in the directory,
      the script should complain.
    \item The test whether files are `the same' should be made with the \n{diff}
      command. Explore options that allow \n{diff} to ignore differences that
      are only in whitespace.
    \end{itemize}
  \end{itemize}
\end{numberedframe}

\begin{numberedframe}{Turn it in!}
  \label{sl-lnx:plag-test}
  Here is how you submit your homework.
  \begin{itemize}
  \item There is a test/submit script:\\
    \n{sds_plagiarism yourscript.sh}\\
    This tests the correctness of your script.
  \item If your script passes the test, use the \n{-s} option to submit:\\
    \n{sds_plagiarism -s yourscript.sh}\\
    or use the \n{-i} option to submit incomplete:\\
    \n{sds_plagiarism -i yourscript.sh}
  \item Add the \n{-d} option for some debugging output:\\
    \n{sds_plagiarism -d yourscript.sh}
  \item (after you run the script once, you'll see in your directory
    the files that are used for testing)
  \end{itemize}
\end{numberedframe}

\end{document}

\begin{numberedframe}{}
  \label{sl-lnx:}
  \begin{itemize}
  \item 
  \end{itemize}
\end{numberedframe}


