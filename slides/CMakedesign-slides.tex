% -*- latex -*-
%%%%%%%%%%%%%%%%%%%%%%%%%%%%%%%%%%%%%%%%%%%%%%%%%%%%%%%%%%%%%%%%
%%%%%%%%%%%%%%%%%%%%%%%%%%%%%%%%%%%%%%%%%%%%%%%%%%%%%%%%%%%%%%%%
%%%%
%%%% This text file is part of the source of slides for
%%%% `The Art of HPC, vol 1: The Science of Computing'
%%%% by Victor Eijkhout, copyright 2012-2024
%%%%
%%%% CMakedesign-slides.tex : cmake slides for making CMakeLists.txt
%%%% see hpc_cmake.tex for driver file
%%%%
%%%%%%%%%%%%%%%%%%%%%%%%%%%%%%%%%%%%%%%%%%%%%%%%%%%%%%%%%%%%%%%%
%%%%%%%%%%%%%%%%%%%%%%%%%%%%%%%%%%%%%%%%%%%%%%%%%%%%%%%%%%%%%%%%

\subsectionframe{Make your own CMake configuration}

\begin{numberedframe}{What are we talking here?}
  You have a code that you want to distribute in source form
  for easy installation.

  You decide to use CMake for portability.

  You think that using CMake might make life easier.

  $\Rightarrow$~To do: write the \n{CMakeLists.txt} file.
\end{numberedframe}

\lstset{language=CMake}
\lstset{style=greencode}

\begin{numberedframe}{The CMakeLists file}
\begin{lstlisting}
cmake_minimum_required( VERSION 3.12 )
project( myproject VERSION 1.0 )
\end{lstlisting}
  \begin{itemize}
  \item Which cmake version is needed for this file?\\
    (CMake has undergone quite some evolution!)
  \item Give a name to your project.
  \item Maybe pick a language.\\
    C~and C++ available by default, or:
\begin{lstlisting}
enable_language(Fortran)
\end{lstlisting}
(list: \lstinline{C}, \lstinline{CXX}, \lstinline{CSharp}, \lstinline{CUDA}, \lstinline{OBJC},
\lstinline{OBJCXX}, \lstinline{Fortran}, \lstinline{HIP}, \lstinline{ISPC}, \lstinline{Swift},
and a couple of variants of \lstinline{ASM})
  \end{itemize}
\end{numberedframe}

\begin{numberedframe}{Target philosophy}
  \begin{itemize}
  \item Declare a target: something that needs to be built,
    and specify what is needed for it
\begin{lstlisting}
add_executable( myprogram )
target_sources( myprogram PRIVATE program.cxx )
\end{lstlisting}
Use of macros:
\begin{lstlisting}
add_executable( ${PROJECT_NAME} )
\end{lstlisting}
\item Do things with the target, for instance state where it
  is to be installed:
\begin{lstlisting}
install( TARGETS myprogram DESTINATION . )
\end{lstlisting}
relative to the prefix location.
  \end{itemize}
\end{numberedframe}

\begin{numberedframe}{Example: single source}
  Build an executable from a single source file:
  
  \lstinputlisting{tutorials/cmake/single/CMakeLists.txt}
\end{numberedframe}

\begin{numberedframe}{Deprecated usage}
\begin{lstlisting}
add_executable( myprogram myprogram.c myprogram.h )
\end{lstlisting}
Prefer `target' design.
\end{numberedframe}

\begin{numberedframe}{Exercise}
  \begin{itemize}
  \item Write a `hello world' program;
  \item Make a CMake setup to compile and install it;
  \item Test it all.
  \end{itemize}
\end{numberedframe}

\begin{numberedframe}{Exercise: using the Eigen library}
This is a short program using Eigen:
\begin{lstlisting}
#include <iostream>
#include <Eigen/Dense>

int main() {
    // Define a 3x3 matrix
    Eigen::Matrix3d matrix;
    matrix << 1, 2, 3,
              4, 5, 6,
              7, 8, 9;
    std::cout << "Original matrix:\n" << matrix << std::endl;
    return 0;
}
\end{lstlisting}
  \begin{itemize}
  \item Make a CMake setup to compile and install it;
  \item Test it.
  \end{itemize}
\end{numberedframe}

\begin{numberedframe}{Make your own library}
First a library that goes into the executable:
\begin{lstlisting}
add_library( auxlib )
target_sources( auxlib PRIVATE aux.cxx aux.h )
target_link_libraries( program PRIVATE auxlib )
\end{lstlisting}
\end{numberedframe}

\begin{numberedframe}{Library during build, setup}
  Full configuration for an executable that uses a library:
  \lstset{numbers=left,numberstyle=\tiny}
  \lstinputlisting{tutorials/cmake/privatelib/CMakeLists.txt}  
  Library shared by default; see later.
\end{numberedframe}

\begin{numberedframe}{Shared and static libraries}
  In the configuration file:
\begin{lstlisting}
add_library( auxlib STATIC )
# or
add_library( auxlib SHARED )
\end{lstlisting}
(default shared if left out),
  or by adding a runtime flag
\begin{lstlisting}
cmake -D BUILD_SHARED_LIBS=TRUE
\end{lstlisting}
Build both by having two lines, one for shared,  one for static.

Related: the \indextermtt{-fPIC} compile option
is set by \indexcmake{CMAKE_POSITION_INDEPENDENT_CODE}:
\begin{lstlisting}
cmake -D CMAKE_POSITION_INDEPENDENT_CODE=ON
\end{lstlisting}
\end{numberedframe}

\begin{numberedframe}{Release a library}
  To have the library released too, use \lstinline{PUBLIC}.\\
  Add the library target to the \lstinline{install} command.
\end{numberedframe}

\begin{numberedframe}{Example: released library}
  \lstset{numbers=left,numberstyle=\tiny}
  \lstinputlisting{tutorials/cmake/withlib/CMakeLists.txt}  
  Note the separate destination directories.
\end{numberedframe}

\begin{numberedframe}{We are getting realistic}
  The previous setup was messy\\
  Better handle the library through a recursive cmake\\
  and make the usual \n{lib} \n{include} \n{bin} setup
\end{numberedframe}

\begin{numberedframe}{Recursive setup, main directory}
  Declare that there is a directory to do recursive make:
  \lstset{numbers=left,numberstyle=\tiny}
  \lstinputlisting{tutorials/cmake/recursive/CMakeLists.txt}
  (Note that the name of the library comes from the subdirectory)
\end{numberedframe}

\begin{numberedframe}{Recursive setup, subdirectory}
  Installs into \n{lib} and \n{include}
  \lstset{numbers=left,numberstyle=\tiny}
  \lstinputlisting{tutorials/cmake/recursive/lib/CMakeLists.txt}
\end{numberedframe}

\begin{numberedframe}{External libraries}
  \begin{itemize}
  \item Use \indexunix{LD_LIBRARY_PATH}, or
  \item use \indexunix{rpath}.
  \end{itemize}
  (Apple note: forced to use second option)
\begin{lstlisting}[basicstyle=\scriptsize\ttfamily]
set_target_properties(
    ${PROGRAM_NAME} PROPERTIES
    BUILD_RPATH   "${CATCH2_LIBRARY_DIRS};${FMTLIB_LIBRARY_DIRS}"
    INSTALL_RPATH "${CATCH2_LIBRARY_DIRS};${FMTLIB_LIBRARY_DIRS}"
    )
\end{lstlisting}
\end{numberedframe}

\endinput

\begin{numberedframe}{}
  \begin{itemize}
  \item 
  \end{itemize}
\begin{lstlisting}
\end{lstlisting}
\end{numberedframe}

