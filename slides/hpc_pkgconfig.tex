%%%%%%%%%%%%%%%%%%%%%%%%%%%%%%%%%%%%%%%%%%%%%%%%%%%%%%%%%%%%%%%%
%%%%%%%%%%%%%%%%%%%%%%%%%%%%%%%%%%%%%%%%%%%%%%%%%%%%%%%%%%%%%%%%
%%%%
%%%% This text file is part of the source of 
%%%% `The Art of HPC, vol 1: The Science of Computing'
%%%% by Victor Eijkhout, copyright 2012-2023
%%%%
%%%% This book is distributed under a Creative Commons Attribution 3.0
%%%% Unported (CC BY 3.0) license and made possible by funding from
%%%% The Saylor Foundation \url{http://www.saylor.org}.
%%%%
%%%% hpc_pkgconfig.tex : Tutorial on pkgconfig, mostly for cmake
%%%%
%%%%%%%%%%%%%%%%%%%%%%%%%%%%%%%%%%%%%%%%%%%%%%%%%%%%%%%%%%%%%%%%
%%%%%%%%%%%%%%%%%%%%%%%%%%%%%%%%%%%%%%%%%%%%%%%%%%%%%%%%%%%%%%%%

\documentclass[10pt]{beamer}

\input courseformat
\def\qrcode{qrvol1}

\begin{document}
\input semester
\title[Pkgconfig tutorial]{The pkgconfig ecosystem}
\author[Eijkhout]{Victor Eijkhout}
\date{\hpcsemester}

\maketitle

\begin{frame}{Justification}
  \n{Pkgconfig} is a \textit{de facto} discovery mechanism
  for CMake packages. We discuss:
  \begin{itemize}
  \item how to discover package
  \item how to make your package discoverable
  \end{itemize}
\end{frame}

\input CMakepc-slides
\input CMakepcin-slides

\end{document}
