% -*- latex -*-
%%%%%%%%%%%%%%%%%%%%%%%%%%%%%%%%%%%%%%%%%%%%%%%%%%%%%%%%%%%%%%%%
%%%%%%%%%%%%%%%%%%%%%%%%%%%%%%%%%%%%%%%%%%%%%%%%%%%%%%%%%%%%%%%%
%%%%
%%%% This text file is part of the source of 
%%%% `Introduction to High-Performance Scientific Computing'
%%%% by Victor Eijkhout, copyright 2012/3/4/5/
%%%%
%%%% This book is distributed under a Creative Commons Attribution 3.0
%%%% Unported (CC BY 3.0) license and made possible by funding from
%%%% The Saylor Foundation \url{http://www.saylor.org}.
%%%%
%%%%
%%%%%%%%%%%%%%%%%%%%%%%%%%%%%%%%%%%%%%%%%%%%%%%%%%%%%%%%%%%%%%%%
%%%%%%%%%%%%%%%%%%%%%%%%%%%%%%%%%%%%%%%%%%%%%%%%%%%%%%%%%%%%%%%%

%\index{complexity|(}

At various places in this book we are interested in how many
operations an algorithm takes. It depends on the context what these
operations are, but often we count additions (or subtractions) and
multiplications. This is called the \emph{arithmetic} or
\indextermsub{computational}{complexity}
of an algorithm. For
instance, summing $n$ numbers takes $n-1$ additions.  Another quantity
that we may want to describe is the amount of space (computer memory)
that is needed. Sometimes the space to fit the input and output of an
algorithm is all that is needed, but some algorithms need temporary
space. The total required space is called the
\indextermsub{space}{complexity}
of an algorithm.

Both arithmetic and space complexity depend on some description of the
input, for instance, for summing an array of numbers, the length $n$
of the array is all that is needed. We express this dependency by
saying `summing an array of numbers has time complexity $n-1$
additions, where $n$ is the length of the array'.

The time (or space) the summing algorithm takes is not dependent on
other factors such as the values of the numbers.  By contrast, some
algorithms such as computing the greatest common divisor of an array
of integers \emph{can} be dependent on the actual values.

\begin{exercise}
  What is the time and space complexity of multiplying two square
  matrices of size $n\times n$? Assume that an addition and a
  multiplication take the same amount of time.
\end{exercise}

Often we aim to simplify the formulas that describe time or space
complexity. For instance, if the complexity of an algorithm is
$n^2+2n$, we see that for $n>2$ the complexity is less than $2n^2$,
and for $n>4$ it is less than $(3/2)n^2$. On the other hand, for all
values of~$n$ the complexity is at least~$n^2$.  Clearly, the
quadratic term $n^2$ is the most important, and the linear term~$n$
becomes less and less important by ratio. We express this informally
by saying that the complexity is quadratic in $n$ as
$n\rightarrow\infty$: there are constants~$c,C$ so that for $n$ large
enough the complexity is at least $cn^2$ and at most~$Cn^2$. 

This is
expressed for short by saying that the complexity is of
\indexterm{order} $n^2$, written as~$O(n^2)$ as
$n\rightarrow\infty$. In
chapter~\ref{ch:odepde} you will see phenomena that we want to
describe as orders of a parameter that goes to zero. In that case we
write for instance $f(h)=O(h^2)$ as $h\downarrow0$, meaning that $f$
is bounded by $ch^2$ and $Ch^2$ for certain constants $c,C$ and $h$
small enough.

\Level 0 {Formal definitions}

Formally, we distinguish the following complexity types:
\begin{itemize}
\item Big-Oh complexity. This states the existence of an upperbound
  on the absolute magnitude:
  \[ g(x) = O(f(x)) \equiv \exists_{C>0,x_0>0} \forall_{x>x_0} \colon |g(x)|<Cf(x) \]
\item Little-oh complexity. This states a function is essentially smaller than another:
  \[ g(x) = o(f(x)) \equiv \forall_{C>0} \exists_{x_0>0} \forall_{x>x_0} \colon |g(x)|<Cf(x) \]
\item Big-Theta complexity. This is like Big-Oh, except that now we have upper and lower bounds:
  \[ g(x) = \Theta(f(x)) \equiv \exists_{C>c>0,x_0>0} \forall_{x>x_0} \colon cf(x) < g(x) < Cf(x) \]
\item Finally: Big-Omega complexity states that a function is at least as large as another:
  \[ g(x) = \Omega(f(x)) \equiv \exists_{C>0,x_0>0} \forall_{x>x_0} \colon g(x) > Cf(x) \]
\end{itemize}

\Level 0 {The Master Theorem}
\label{sec:compl-master}

For many operations, complexity is easily expressed recursively.
For instance, the \indextermsub{complexity of}{quicksort}
obeys a recurrence
\[ T(n) = 2T(n/2) + O(n), \]
expressing that sorting an array of $n$ numbers involves twice sorting $n/2$ numbers,
plus the splitting step, which is linear in the array size.

The so-called \indextermdef{master theorem} of complexity
allows you to convert these recursive expression to a closed form formula.

Suppose the function~$T$ satisfies
\[ T(n) = a T\bigl( \frac{n}{b} \bigr) + f(n) \]
with $a,b\geq 1$ and $f(\cdot)$ asymptotically positive.
Then there are three cases:
\begin{enumerate}
\item If
  \[ f(n)=O\left( n^{\log_b(a)-\epsilon} \right )
  \quad\hbox{then}\quad
  T(n) = \Theta \left( n^{\log_b(a)} \right )
  \]
\item If
  \[ f(n)=O\left( n^{\log_b(a)} \right )
  \quad\hbox{then}\quad
  T(n) = \Theta \left( n^{\log_b(a)}\log(n) \right )
  \]
\item If
  \[ f(n)=O\left( n^{\log_b(a)+\epsilon} \right )
  \quad\hbox{then}\quad
  T(n) = \Theta \left( f(n) \right )
  \]
\end{enumerate}

Some examples follow.

\Level 1 {Quicksort and mergesort}

The \indexterm{quicksort} and \indexterm{mergesort} algorithms
recursively divide work into two (hopefully) equal halves,
preceeded/followed by a linear splitting/merging step.
Thus their complexity satisfies
\[ T(n) = 2T(n/2) + O(n) \]
giving, with $a=b=2$ and $f(n)=n$, by case 2:
\[ T(n) = \Theta( n\log n ). \]

\Level 1 {Matrix-matrix multiplication}

We already know that the ordinary triple-loop formulation of
matrix-matrix multiplication has a complexity of~$O(n^3)$.
Multiplying matrices by recursive bisection into $2\times 2$ blocks
gives
\[ T(n) = 8T(n/2) + O(n^2). \]
Substituting $a=8,b=2$ gives $\log_b(a)=3$
so case~1 applies with $\epsilon=1$,
and we find
\[ T(n) = \Theta( n^3 ). \]

Slightly more interesting, the
\indexterm{Strassen}\index{Strassen algorithm|see{matrix, times matrix product, Strassen algorithm for}}
algorithm~\cite{St:gaussnotoptimal}
has $a=7$,
giving an essentially lower complexity of~$O(n^{\log 7}) \approx O(n^{2.81})$.
There are other algorithms like this~\cite{Pa:combinations},
with slightly lower complexities,
though all still well above~$O(n^2)$.

%\index{complexity|)}
