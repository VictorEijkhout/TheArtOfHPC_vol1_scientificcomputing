% -*- latex -*-
%%%%%%%%%%%%%%%%%%%%%%%%%%%%%%%%%%%%%%%%%%%%%%%%%%%%%%%%%%%%%%%%
%%%%%%%%%%%%%%%%%%%%%%%%%%%%%%%%%%%%%%%%%%%%%%%%%%%%%%%%%%%%%%%%
%%%%
%%%% This text file is part of the source of 
%%%% `Introduction to High-Performance Scientific Computing'
%%%% by Victor Eijkhout, copyright 2012/3/4/5/
%%%%
%%%% This book is distributed under a Creative Commons Attribution 3.0
%%%% Unported (CC BY 3.0) license and made possible by funding from
%%%% The Saylor Foundation \url{http://www.saylor.org}.
%%%%
%%%%
%%%%%%%%%%%%%%%%%%%%%%%%%%%%%%%%%%%%%%%%%%%%%%%%%%%%%%%%%%%%%%%%
%%%%%%%%%%%%%%%%%%%%%%%%%%%%%%%%%%%%%%%%%%%%%%%%%%%%%%%%%%%%%%%%

Taylor series expansion is a powerful mathematical tool. In this
course it is used several times in proving properties of numerical
methods.

The Taylor expansion theorem, in a sense, asks how well functions can
be approximated by polynomials, that is, for a given
function~$f$, can we find coefficients~$c_i$ with $i=1,\ldots,n$ so that
\[ f(x)\approx c_0+c_1x+c_2x^2+\cdots+c_nx^n.\]
This question obviously needs to be refined. What do we mean by
`approximately equal'? This approximation formula can not
hold for all functions~$f$ and all~$x$: the function $\sin x$ is
bounded for all~$x$, but any polynomial is unbounded for $x\rightarrow
\pm\infty$, so any polynomial approximation to the $\sin x$ function
is unbounded. Clearly we can only approximate on an interval.

We will show that a function~$f$ with sufficiently many derivatives
can be approximated as follows: if the $n$-th derivative~$f^{(n)}$ is
continuous on an interval~$I$, then there are coefficients
$c_0,\ldots,c_{n-1}$ such that
\[
  \forall_{x\in I}\colon |f(x)-\sum_{i<n}c_ix^i|\leq cM_n
    \qquad\hbox{where $M_n=\max_{x\in I}|f^{(n)}(x)|$}
\]
It is easy to get inspiration for what these coefficients should
be. Suppose
\[ f(x) = c_0+c_1x+c_2x^2+\cdots \]
(where we will not worry about matters of convergence and how long the
dots go on) then filling in
\[ \hbox{$x=0$ gives $c_0=f(0)$.} \]
Next, taking the first derivative 
\[ f'(x) = c_1+2c_2x+3c_3x^2+\cdots \]
and filling in
\[ \hbox{$x=0$ gives $c_1=f'(0)$.} \]
From the second derivative
\[ f''(x) = 2c_2+6c_3x+\cdots \]
so filling in $x=0$ gives
\[ c_2=f''(0)/2. \]
Similarly, in the third derivative
\[ \hbox{filling in $x=0$ gives $c_3=\frac1{3!}f^{(3)}(0)$.} \]

Now we need to be a bit more precise. Cauchy's form of Taylor's
theorem says that
\[ f(x) = 
    f(a)+\frac1{1!}f'(a)(x-a)+\cdots+\frac1{n!}f^{(n)}(a)(x-a)^n
    +R_n(x)
\]
where the `rest term' $R_n$ is 
\[ R_n(x) = \frac1{(n+1)!}f^{(n+1)}(\xi)(x-a)^{n+1}
    \quad\hbox{where $\xi\in(a,x)$ or $\xi\in(x,a)$ depending.}
\]
If $f^{(n+1)}$ is bounded, and $x=a+h$, then the form in which we
often use
Taylor's theorem is 
\[ f(x) = \sum_{k=0}^n \frac1{k!}f^{(k)}(a)h^k+O(h^{n+1}).\]
We have now approximated the function~$f$ on a certain interval by a
polynomial, with an error that decreases geometrically with the
inverse of the degree of the polynomial.

For a proof of Taylor's theorem we use integration by parts. First we
write
\[ \int_a^xf'(t)dt=f(x)-f(a) \]
as
\[ f(x) = f(a)+\int_a^xf'(t)dt \]
Integration by parts then gives
\[
\begin{array}{r@{=}l}
  f(x)
    &f(a)+[xf'(x)-af'(a)]-\int_a^xtf''(t)dt\\
    &f(a)+[xf'(x)-xf'(a)+xf'(a)-af'(a)]-\int_a^xtf''(t)dt\\
    &f(a)+x\int_a^xf''(t)dt+(x-a)f'(a)-\int_a^xtf''(t)dt\\
    &f(a)+(x-a)f'(a)+\int_a^x(x-t)f''(t)dt\\
\end{array}
\]
Another application of integration by parts gives
\[ f(x)=f(a)+(x-a)f'(a)+\frac12(x-a)^2f''(a)
    +\frac12 \int_a^x(x-t)^2f'''(t)dt
\]
Inductively, this gives us Taylor's theorem with
\[ R_{n+1}(x) = \frac1{n!}\int_a^x(x-t)^nf^{(n+1)}(t)dt \]
By the mean value theorem this is
\[
\begin{array}{r@{=}l}
R_{n+1}(x)
&\frac1{(n+1)!}f^{(n+1)}(\xi)\int_a^x(x-t)^nf^{(n+1)}(t)dt\\
&\frac1{(n+1)!}f^{(n+1)}(\xi)(x-a)^{n+1}
\end{array}
\]

