% -*- latex -*-
%%%%%%%%%%%%%%%%%%%%%%%%%%%%%%%%%%%%%%%%%%%%%%%%%%%%%%%%%%%%%%%%
%%%%%%%%%%%%%%%%%%%%%%%%%%%%%%%%%%%%%%%%%%%%%%%%%%%%%%%%%%%%%%%%
%%%%
%%%% This text file is part of the source of 
%%%% `Introduction to High-Performance Scientific Computing'
%%%% by Victor Eijkhout, copyright 2012/3/4/5/
%%%%
%%%% This book is distributed under a Creative Commons Attribution 3.0
%%%% Unported (CC BY 3.0) license and made possible by funding from
%%%% The Saylor Foundation \url{http://www.saylor.org}.
%%%%
%%%%
%%%%%%%%%%%%%%%%%%%%%%%%%%%%%%%%%%%%%%%%%%%%%%%%%%%%%%%%%%%%%%%%
%%%%%%%%%%%%%%%%%%%%%%%%%%%%%%%%%%%%%%%%%%%%%%%%%%%%%%%%%%%%%%%%

Partial Differential Equations\index{partial differential equations}
are the source of a large fraction of \ac{HPC} problems. Here is a
quick derivation of two of the most important ones.

\Level 0 {Partial derivatives}

Derivatives of a function $u(x)$ are a measure of the rate of
change. Partial derivatives to the same, but for a function $u(x,y)$
of two variables. Notated $u_x$ and $u_y$, these \indexterm{partial
  derivates} indicate the rate of change if only one variable changes
and the other stays constant.

Formally, we define
$u_x,u_y$ by:
\[ u_x(x,y) = \lim_{h\rightarrow0}\frac{u(x+h,y)-u(x,y)}h,\quad
   u_y(x,y) = \lim_{h\rightarrow0}\frac{u(x,y+h)-u(x,y)}h
\]

\Level 0 {Poisson or Laplace Equation}

Let $T$ be the temperature of a material, then its heat energy is
proportional to it. A~segment of length~$\Delta x$ has heat energy
$Q=c\Delta x\cdot u$. If the heat energy in that
segment is constant
\[ \frac{\delta Q}{\delta t}=c\Delta x\frac{\delta u}{\delta t}=0 \]
but it is also the difference between inflow and outflow of the
segment. Since flow is proportional to temperature differences, that
is, to~$u_x$, we see that also
\[ 0=
    \left.\frac{\delta u}{\delta x}\right|_{x+\Delta x}-
    \left.\frac{\delta u}{\delta x}\right|_{x}
\]
In the limit of $\Delta x\downarrow0$ this gives $u_{xx}=0$, which is
called the \indexterm{Laplace equation}. If we have a source term, for
instance corresponding to externally applied heat, the equation
becomes $u_{xx}=f$, which is called the \indexterm{Poisson equation}.

\Level 0 {Heat Equation}
\label{sec:derive-heat}

Let $T$ be the temperature of a material, then its heat energy is
proportional to it. A~segment of length~$\Delta x$ has heat energy
$Q=c\Delta x\cdot u$. The rate of change in heat energy in that
segment is
\[ \frac{\delta Q}{\delta t}=c\Delta x\frac{\delta u}{\delta t} \]
but it is also the difference between inflow and outflow of the
segment. Since flow is proportional to temperature differences, that
is, to~$u_x$, we see that also
\[ \frac{\delta Q}{\delta t}=
    \left.\frac{\delta u}{\delta x}\right|_{x+\Delta x}-
    \left.\frac{\delta u}{\delta x}\right|_{x}
\]
In the limit of $\Delta x\downarrow0$ this gives $u_t=\alpha u_{xx}$.

\Level 0 {Steady state}
\label{app:steadystate}

The solution of an \ac{IBVP} is a function $u(x,t)$. In cases where
the forcing function and the boundary conditions do not depend on
time, the solution will converge in time, to a function called the
\indexterm{steady state} solution:
\[ \lim_{t\rightarrow\infty} u(x,t)=u_{\mathrm{steady state}}(x). \]
This solution satisfies a \ac{BVP}, which can be found by setting
$u_t\equiv\nobreak0$. For instance, for the heat equation \[
u_t=u_{xx}+q(x) \] the steady state solution satisfies $-u_{xx}=q(x)$.

