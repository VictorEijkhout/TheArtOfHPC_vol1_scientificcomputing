%%%%%%%%%%%%%%%%%%%%%%%%%%%%%%%%%%%%%%%%%%%%%%%%%%%%%%%%%%%%%%%%
%%%%%%%%%%%%%%%%%%%%%%%%%%%%%%%%%%%%%%%%%%%%%%%%%%%%%%%%%%%%%%%%
%%%%
%%%% This text file is part of the source of 
%%%% `Introduction to High-Performance Scientific Computing'
%%%% by Victor Eijkhout, copyright 2012-7
%%%%
%%%% This book is distributed under a Creative Commons Attribution 3.0
%%%% Unported (CC BY 3.0) license and made possible by funding from
%%%% The Saylor Foundation \url{http://www.saylor.org}.
%%%%
%%%%
%%%%%%%%%%%%%%%%%%%%%%%%%%%%%%%%%%%%%%%%%%%%%%%%%%%%%%%%%%%%%%%%
%%%%%%%%%%%%%%%%%%%%%%%%%%%%%%%%%%%%%%%%%%%%%%%%%%%%%%%%%%%%%%%%

You have now seen several computing models: single core, shared memory
multicore, distributed memory clusters, GPUs. These models all have in
common that, if there is more than one instruction stream active, all
streams are interchangeable. With regard to GPUs we need to refine this
statement: all instruction stream \emph{on the GPU} are
interchangeable. However, a GPU is not a standalone device, but can be
considered a \indexterm{co-processor} to a \indexterm{host
  processor}. 

If we want to let the host perform useful work while the co-processor
is active, we now have two different instruction streams or types of
streams. This situation is known as \emph{heterogeneous computing}.
In the GPU case, these instruction streams are even
programmed by a slightly different mechanisms --~using
\indexterm{CUDA} for the GPU~-- but this need not be the case: the
\index{Intel!MIC} Intel \acf{MIC} architecture is programmed in
ordinary~C.

% LocalWords:  Eijkhout multicore GPUs GPU standalone co CUDA MIC
