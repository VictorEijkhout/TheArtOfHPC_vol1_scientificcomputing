%%%%%%%%%%%%%%%%%%%%%%%%%%%%%%%%%%%%%%%%%%%%%%%%%%%%%%%%%%%%%%%%
%%%%%%%%%%%%%%%%%%%%%%%%%%%%%%%%%%%%%%%%%%%%%%%%%%%%%%%%%%%%%%%%
%%%%
%%%% This text file is part of the source of 
%%%% `The Art of HPC, vol 1: The Science of Computing'
%%%% by Victor Eijkhout, copyright 2012
%%%%
%%%% This book is distributed under a Creative Commons Attribution 3.0
%%%% Unported (CC BY 3.0) license and made possible by funding from
%%%% The Saylor Foundation \url{http://www.saylor.org}.
%%%%
%%%%
%%%%%%%%%%%%%%%%%%%%%%%%%%%%%%%%%%%%%%%%%%%%%%%%%%%%%%%%%%%%%%%%
%%%%%%%%%%%%%%%%%%%%%%%%%%%%%%%%%%%%%%%%%%%%%%%%%%%%%%%%%%%%%%%%

\acfp{FPGA} are reconfigurable devices: the electronics are organized
in low level building block whose function can be changed
dynamically. Even the connections between the building blocks can be
changed. Thus, \acp{FPGA} are suitable for experimenting with
hardware before an actual chip is built; see
\url{http://ramp.eecs.berkeley.edu/} for one example.

The fact that \acp{FPGA} are typically much slower than regular CPUs
is not a major concern when they are used for prototyping.  However,
they are in fact suitable for certain calculations. Since the
computation is `hard-wired' in an \ac{FPGA}, no silicon is wasted,
making the device very energy efficient. Not all computations are
candidates for realization in an \ac{FPGA}, but as one example they
are popular for certain calculations in \indexterm{computational
  finance}.


% LocalWords:  Eijkhout FPGA
