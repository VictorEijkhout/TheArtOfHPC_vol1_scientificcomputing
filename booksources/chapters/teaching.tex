% -*- latex -*-
%%%%%%%%%%%%%%%%%%%%%%%%%%%%%%%%%%%%%%%%%%%%%%%%%%%%%%%%%%%%%%%%
%%%%%%%%%%%%%%%%%%%%%%%%%%%%%%%%%%%%%%%%%%%%%%%%%%%%%%%%%%%%%%%%
%%%%
%%%% This text file is part of the source of 
%%%% `Introduction to High-Performance Scientific Computing'
%%%% by Victor Eijkhout, copyright 2012-2021
%%%%
%%%% This book is distributed under a Creative Commons Attribution 3.0
%%%% Unported (CC BY 3.0) license and made possible by funding from
%%%% The Saylor Foundation \url{http://www.saylor.org}.
%%%%
%%%%%%%%%%%%%%%%%%%%%%%%%%%%%%%%%%%%%%%%%%%%%%%%%%%%%%%%%%%%%%%%
%%%%%%%%%%%%%%%%%%%%%%%%%%%%%%%%%%%%%%%%%%%%%%%%%%%%%%%%%%%%%%%%

The material in this book is far more than will fit a one-semester course.
Here are a couple of strategies for how to teach it as a course,
or incorporate it in an other course.

\Level 0 {Standalone course}

\Level 0 {Addendum to parallel programming class}

Suggested teachings and exercises:

Parallel analysis:
\ref{ex:amdahl-e}, \ref{ex:gustaf-e}, \ref{ex:brent-tree}.

Architecture:
\ref{ex:p-d-D-relation}, \ref{ex:linear-bisection},
\ref{ex:bisection-cube}, \ref{ex:bisection-hypercube}.

Complexity: 
\ref{ex:sum-hypercube}

\Level 0 {Tutorials}


% LocalWords:  Eijkhout Standalone
