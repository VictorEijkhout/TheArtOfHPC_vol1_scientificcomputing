%%%%%%%%%%%%%%%%%%%%%%%%%%%%%%%%%%%%%%%%%%%%%%%%%%%%%%%%%%%%%%%%
%%%%%%%%%%%%%%%%%%%%%%%%%%%%%%%%%%%%%%%%%%%%%%%%%%%%%%%%%%%%%%%%
%%%%
%%%% This text file is part of the source of 
%%%% `Introduction to High-Performance Scientific Computing'
%%%% by Victor Eijkhout, copyright 2012-2022
%%%%
%%%% This book is distributed under a Creative Commons Attribution 3.0
%%%% Unported (CC BY 3.0) license and made possible by funding from
%%%% The Saylor Foundation \url{http://www.saylor.org}.
%%%%
%%%%
%%%%%%%%%%%%%%%%%%%%%%%%%%%%%%%%%%%%%%%%%%%%%%%%%%%%%%%%%%%%%%%%
%%%%%%%%%%%%%%%%%%%%%%%%%%%%%%%%%%%%%%%%%%%%%%%%%%%%%%%%%%%%%%%%

A~good part of being an effective practitioner
of High Performance Scientific Computing is
what can be called `HPC Carpentry':
a~number of skills that are not scientific in nature,
but that are still indispensable to getting your work done.

The vast majority of
scientific programming is done on the Unix platform so we start out
with a tutorial on Unix in chapter~\ref{tut:unix}, followed by an
explanation of the how your code is handled by compilers and linkers
and such in chapter~\ref{tut:compile}.

Next you will learn about some tools that will increase your
productivity and effectiveness: 
\begin{itemize}
\item The \emph{Make} utility is used for managing the building of
  projects; chapter~\ref{tut:gnumake}.
\item Source control systems store your code in such a way that you
  can undo changes, or maintain multiple versions; in
  chapter~\ref{tut:git} you will see the \emph{subversion} software.
\item Storing and exchanging scientific data becomes an important
  matter once your program starts to produce results; in
  chapter~\ref{tut:hdf5} you will learn the use of \emph{HDF5}.
\item Visual output of program data is important, but too wide a topic
  to discuss here in great detail; chapter~\ref{tut:gnuplot} teaches
  you the basics of the \emph{gnuplot} package, which is suitable for
  simple data plotting.
\end{itemize}

We also consider the activity of program development itself:
chapter~\ref{tut:coding} considers how to code to prevent errors, and
chapter~\ref{tut:debug} teaches you to debug code with
\emph{gdb}. 
\begin{notready}
Chapter~\ref{tut:performance} discusses measuring the
performance of code.
\end{notready}
Chapter~\ref{tut:language} contains some information on
how to write a program that uses more than one programming language.

Finally, chapter~\ref{tut:latex} teaches you about the \LaTeX{}
document system, so that you can report on your work in beautifully
typeset articles.

Many of the tutorials are very hands-on. Do them while sitting at a
computer!

\begin{table}[t]
  \setcounter{lesson}{0}
  \rightskip=0pt plus 1fil\relax
  \setbox0=\hbox{homework }\edef\colwidth{.7in}%{\the\wd0}
  % using array, which is incompatible somehow: >{\raggedright\arraybackslash}
  \begin{tabular}{lp{\colwidth}p{\colwidth}p{\colwidth}p{\colwidth}p{\colwidth}p{\colwidth}}
    \toprule % \hline
    &&&&\multicolumn{2}{c}{Exercises}\\
    \cmidrule{5-6}
    lesson&Topic&Book&Slides&in-class&homework\\
    \midrule % \hline

    %%
    \stepcounter{lesson}\arabic{lesson}
    &Unix &\ref{tut:unix} &unix&&\ref{tut:ex:plagiarism}\\

    %%
    \stepcounter{lesson}\arabic{lesson}
    &Git &\ref{tut:git} &\\

    %%
    \stepcounter{lesson}\arabic{lesson}
    &Programming &\ref{tut:compile} &programming&\ref{ex:compile3}&\ref{ex:givens-optimize}\\

    %%
    \stepcounter{lesson}\arabic{lesson}
    &Libraries &\ref{tut:compile} &programming\\

    %%
    \stepcounter{lesson}\arabic{lesson}
    &Debugging &\ref{tut:debug} &&&root code\\

    %%
    \stepcounter{lesson}\arabic{lesson}
    &\LaTeX{} &\ref{tut:latex} &&&\ref{ex:latex-doc}\\

    %%
    \stepcounter{lesson}\arabic{lesson}
    &Make &\ref{tut:gnumake} &&&\ref{ex:make-main-lib}, \ref{ex:make-main-lib-f}\\

    \bottomrule % \hline
  \end{tabular}
\caption{Timetable for the carpentry section of an HPC course.}
\label{tab:carpentry}
\end{table}

Table~\ref{tab:carpentry} gives a proposed lesson outline for the carpentry section
of a course.
The article by Wilson~\cite{Wilson:bestpractices} is a good read
on the thinking behind this `HPC carpentry'.

% LocalWords:  Eijkhout HPC HDF gnuplot gdb
