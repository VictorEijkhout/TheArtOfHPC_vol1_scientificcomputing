% -*- latex -*-
%%%%%%%%%%%%%%%%%%%%%%%%%%%%%%%%%%%%%%%%%%%%%%%%%%%%%%%%%%%%%%%%
%%%%%%%%%%%%%%%%%%%%%%%%%%%%%%%%%%%%%%%%%%%%%%%%%%%%%%%%%%%%%%%%
%%%%
%%%% This text file is part of the source of 
%%%% `The Art of HPC, vol 4: HPC Carpentry'
%%%% by Victor Eijkhout, copyright 2012-2024
%%%%
%%%% This book is distributed under a Creative Commons Attribution 3.0
%%%% Unported (CC BY 3.0) license and made possible by funding from
%%%% The Saylor Foundation \url{http://www.saylor.org}.
%%%%
%%%%%%%%%%%%%%%%%%%%%%%%%%%%%%%%%%%%%%%%%%%%%%%%%%%%%%%%%%%%%%%%
%%%%%%%%%%%%%%%%%%%%%%%%%%%%%%%%%%%%%%%%%%%%%%%%%%%%%%%%%%%%%%%%

\lstset{language=CMake}

\begin{table}[p]
  \catcode`\_=12
  \begin{tabular}{>{\ttfamily}rp{3in}}
    \toprule
    \multicolumn{2}{c}{General directives}\\
    cmake_minimum_required     &specify minimum cmake version\\
    project                    &name and version number of this project\\
    install                    &specify directory where to install targets\\
    \midrule
    \multicolumn{2}{c}{Project building directives}\\
    add_executable             &specify executable name \\
    add_library                &specify library name \\
    add_subdirectory           &specify subdirectory where cmake also needs to run\\
    target_sources             &specify sources for a target\\
    target_link_libraries      &specify executable and libraries to link into it\\
    target_include_directories &specify include directories, privately or publicly\\
    find_package               &other package to use in this build\\
    \midrule
    \multicolumn{2}{c}{Utility stuff}\\
    target_compile_options     &literal options to include\\
    target_compile_features    &things that will be translated by cmake into options\\
    target_compile_definitions &macro definitions to be set private or publicly\\
    file                       &define macro as file list\\
    message                    &Diagnostic to print, subject to level specification\\
    \midrule
    \multicolumn{2}{c}{Control}\\
    if() else() endif()        &conditional\\
    \bottomrule
  \end{tabular}
  \caption{Cmake commands.}
  \label{tab:cmake-commands}
\end{table}

\Level 0 {CMake as build system}

\indexterm{CMake} is a general build system that uses other systems
such as \indexterm{Make} as a back-end.
The general workflow is:
\begin{enumerate}
\item The configuration stage. Here the \indextermtt{CMakeLists.txt} file
  is parsed, and a \n{build} directory populated. This typically looks like:
\begin{lstlisting}[language=bash]
mkdir build
cd build
cmake <source location>
\end{lstlisting}
Some people create the build directory in the source tree,
in which case the \indexterm{CMake} command is
\begin{lstlisting}[language=bash]
cmake ..
\end{lstlisting}
Others put the build directory next to the source, in which case:
\begin{lstlisting}[language=bash]
cmake ../src_directory
\end{lstlisting}
\item The build stage. Here the installation-specific compilation
  in the build directory is performed.
  With \indexterm{Make} as the `generator' this would be
\begin{lstlisting}[language=bash]
cd build
make
\end{lstlisting}
but more generally
\begin{lstlisting}[language=bash]
cmake --build <build directory>
\end{lstlisting}
Alternatively, you could use generators such as \indexterm{ninja},
\indexterm{Visual Studio}, or \indexterm{XCode}:
\begin{lstlisting}[language=bash]
cmake -G ninja
  ## the usual arguments
\end{lstlisting}
\item The install stage. This can move binary files to a permanent
  location, such as putting library files in \n{/usr/lib}:
\begin{lstlisting}[language=bash]
make install
\end{lstlisting}
or 
\begin{lstlisting}[language=bash]
cmake --install <build directory>
\end{lstlisting}
However, the install location already has to be set in the configuration stage.
We will see later in detail how this is done.
\end{enumerate}

Summarizing, the out-of-source workflow as advocated in this tutorial is
\begin{lstlisting}[language=bash]
ls some_package_1.0.0 # we are outside the source
ls some_package_1.0.0/CMakeLists.txt # source contains cmake file
mkdir builddir && cd builddir # goto build location
cmake -D CMAKE_INSTALL_PREFIX=../installdir \
      ../some_package_1.0.0
make
make install
\end{lstlisting}

The resulting directory structure is illustrated in figure~\ref{fig:cmake-buildinstall}.

\begin{figure}[ht]
  \hbox to \textwidth\bgroup
  \begin{minipage}{.45\textwidth}
    \tikzsetnextfilename{cmake-insource}
    \begin{tikzpicture}[dirtree]
      \node{dir}
      child { node {src} child { node {build} }
      }
      child { node {install} }
      ;
    \end{tikzpicture}
  \end{minipage}
  \hss
  \begin{minipage}{.45\textwidth}
    \tikzsetnextfilename{cmake-outsource}
    \begin{tikzpicture}[dirtree]
      \node{dir}
      child { node {src} }
      child { node {build} }
      child { node {install} }
      ;
    \end{tikzpicture}
  \end{minipage}
  \egroup
  \caption{In-source (left) and out-of-source (right) build schemes.}
  \label{fig:cmake-buildinstall}
\end{figure}

\Level 1 {Target philosophy}

Modern \indexterm{CMake} works through declaring targets and their requirements.
Targets are things that will be built, and requirements can be such things as
\begin{itemize}
\item source files
\item include paths and other paths
\item compiler and linker options.
\end{itemize}
There is a distinction between build requirements,
the paths/files/options used during building,
and usage requirements,
the paths/files/options that need to be in force
when your targets are being used.

For requirements during building:
\begin{lstlisting}
target_some_requirement( <target> PRIVATE <requirements> )
\end{lstlisting}
Usage requirements:
\begin{lstlisting}
target_some_requirement( <target> INTERFACE <requirements> )
\end{lstlisting}
Combined:
\begin{lstlisting}
target_some_requirement( <target> PUBLIC <requirements> )
\end{lstlisting}
For future reference, requirements are specified with
\indexcmake{target_include_directories},
\indexcmake{target_compile_definitions},
\indexcmake{target_compile_options},
\indexcmake{target_compile_features},
\indexcmake{target_sources},
\indexcmake{target_link_libraries},
\indexcmake{target_link_options},
\indexcmake{target_link_directories}.

\Level 1 {Languages}

\indexterm{CMake} is largely aimed at~C++, but it easily supports C as well.
For \emph{Fortran}\index{CMake!Fortran support} support,
first do
\begin{lstlisting}
enable_language(Fortran)
\end{lstlisting}
Note that capitalization: this also holds for all variables
such as \indexcmake{CMAKE_Fortran_COMPILER}.

\Level 1 {Script structure}

\begin{tabular}{lp{3in}}
  \toprule
  Commands learned in this section\\
  \midrule
  \lstinline+cmake_minimum_required+&declare minimum required version for this script\\
  \lstinline+project+&declare a name for this project\\
  \bottomrule
\end{tabular}

\indexterm{CMake} is driven by the \indextermttdef{CMakeLists.txt} file.
This needs to be in the root directory of your project.
(You can additionally have files by that name in subdirectories.)

Since \indexterm{CMake} has changed quite a bit over the years,
and is still evolving,
it is a good idea to start each script with a declaration
of the (minimum) required version:
\begin{lstlisting}
cmake_minimum_required( VERSION 3.12 )
\end{lstlisting}
You can query the version of your \indexterm{CMake} executable:
\begin{lstlisting}[language=bash]
$ cmake --version
cmake version 3.19.2
\end{lstlisting}

You also need to declare a project name and version,
which need not correspond to any file names:
\begin{lstlisting}
project( myproject VERSION 1.0 )
\end{lstlisting}

\newpage
\Level 0 {Examples cases}

\Level 1 {Executable from sources}

(The files for this examples are in \n{tutorials/cmake/single}.)

\begin{tabular}{lp{3in}}
  \toprule
  Commands learned in this section\\
  \midrule
  \lstinline+add_executable+&declare an executable as target\\
  \lstinline+target_sources+&specify sources for a target\\
  \lstinline+install+&indicate location where to install this project\\
  \lstinline+PROJECT_NAME+&macro that expands to the project name \\
  \bottomrule
\end{tabular}

\begin{floatingfigure}[r]{2.5in}
  \begin{minipage}{2.5in}
    \tikzsetnextfilename{cmake-install}
    \begin{tikzpicture}[dirtree]
      \node{dir}
      child { node {src}
        child { node {CMakeLists.txt} }
        child { node {program.cxx} } }
      child { node {build}
        child { node {CMakeCache.txt} }
        child { node {Makefile} }
        child { node {myprogram} }
        child { node {lots more} } }
      child { node {install}
        child { node {myprogram} } }
      ;
    \end{tikzpicture}
  \end{minipage}
\end{floatingfigure}
%
If you have a project that is supposed to deliver an executable,
you declare in your \texttt{CMakeLists.txt}:
\begin{lstlisting}
add_executable( myprogram )
target_sources( myprogram PRIVATE program.cxx )
\end{lstlisting}
Often, the name of the executable is the name of the project,
so you'd specify:
\begin{lstlisting}
add_executable( ${PROJECT_NAME} )
\end{lstlisting}
%
\begin{remark}
  An older usage is to specify the sources files directly
  with the executable. The above usage is preferred.
\begin{lstlisting}
add_executable( myprogram program.cxx )
\end{lstlisting}  
\end{remark}

In order to move the executable to the install location,
you need a clause
\begin{lstlisting}
install( TARGETS myprogram DESTINATION . )
\end{lstlisting}
Without the \indexcmake{DESTINATION} clause, a default \n{bin} directory
will be created; specifying \indexcmake{DESTINATION foo} will put the
target in a \n{foo} sub-directory of the installation directory.

\begin{floatingfigure}[r]{2.5in}\begin{minipage}{2.5in}\hbox{}\vskip 4in\hbox{}\end{minipage}\end{floatingfigure}%pyskip
%
In the figure on the right we have also indicated the \n{build} directory,
which from now on we will not show again.
It contains automatically generated files that are hard to
decyper, or debug. Yes, there is a \n{Makefile}, but even for simple
projects this is too complicated to debug by hand if your \indexterm{CMake}
installation misbehaves.

Here is the full \n{CMakeLists.txt}:
%
\lstinputlisting{tutorials/cmake/single/CMakeLists.txt}

\newpage
\Level 1 {Making libraries}

\begin{tabular}{lp{3in}}
  \toprule
  Commands learned in this section\\
  \midrule
  \lstinline+target_include_directories+&indicate location of include files\\
  \lstinline+add_library+&declare a library and its sources\\
  \lstinline+target_link_libraries+&indicate that the library belong with an executable\\
  \bottomrule
\end{tabular}

\Level 2 {Multiple files}

(The files for this example are in \n{tutorials/cmake/multiple}.)

\begin{floatingfigure}[r]{2.5in}
  \begin{minipage}{2.5in}
    \tikzsetnextfilename{cmake-lib}
    \begin{tikzpicture}[dirtree]
      \node{dir}
      child { node {src}
        child { node {CMakeLists.txt} }
        child { node {program.cxx} }
        child { node {lib}
          child { node {aux.cxx} }
          child { node {aux.h} }
          }
      }
      child { node {install}
        child { node {program} }
      }
      ;
    \end{tikzpicture}
%%       [dir [src [CMakeLists.txt] [program.cxx] [aux.cxx] [aux.h] ]
%%         [install [program] ] ]
  \end{minipage}
\end{floatingfigure}
%
If there is only one source file, the previous section is all you need.
If there are multiple files, you could write
\begin{lstlisting}
add_executable( program )
target_sources( program PRIVATE program.cxx aux.cxx aux.h )
\end{lstlisting}
You can also put the non-main source files in a separate directory:
\begin{lstlisting}
add_executable( program )
target_sources( program PRIVATE program.cxx lib/aux.cxx lib/aux.h )
\end{lstlisting}

However, often you will build libraries.
We start by making a \n{lib} directory and indicating that header files
need to be found there:
\begin{lstlisting}
target_include_directories( program PRIVATE lib )
\end{lstlisting}

The actual library is declared with an \indexcmake{add_library} clause:
\begin{lstlisting}
add_library( auxlib )
target_sources( auxlib PRIVATE aux.cxx aux.h )
\end{lstlisting}
Next, you need to link that library into the program:
\begin{lstlisting}
target_link_libraries( program PRIVATE auxlib )
\end{lstlisting}
The \indexcmake{PRIVATE} clause means that the library is only for
purposes of building the executable.
(Use \indexcmake{PUBLIC} to have the library be included in the installation;
we will explore that in section~\ref{sec:cmake-public-lib}.)

The full \n{CMakeLists.txt}:
%
\lstinputlisting{tutorials/cmake/multiple/CMakeLists.txt}

Note that private shared libraries make no sense, as they will give
runtime unresolved references.

\Level 2 {Testing the generated makefiles}

In the Make tutorial \ref{tut:gnumake} you learned how Make will only recompile
the strictly necessary files when a limited edit has been made.
The makefiles generated by \indexterm{CMake} behave similarly.
With the structure above, we first touch the \n{aux.cxx} file,
which necessitates rebuilding the library:
%
\verbatiminput{tutorials/cmake/multiple-touch1.txt}

On the other hand, if we edit a header file, the main program
needs to be recompiled too:
%
\verbatiminput{tutorials/cmake/multiple-touch1.txt}

\newpage
\Level 2 {Making a library for release}
\label{sec:cmake-public-lib}

(The files for this example are in \n{tutorials/cmake/withlib}.)

\begin{tabular}{lp{3in}}
  \toprule
  Commands learned in this section\\
  \midrule
  \lstinline+SHARED+&indicated to make shared libraries\\
  \bottomrule
\end{tabular}

\begin{floatingfigure}[r]{2.5in}
  \begin{minipage}{2.5in}
    \tikzsetnextfilename{cmake-withlib}
    \begin{tikzpicture}[dirtree]
      \node{dir}
      child { node {src}
        child { node {CMakeLists.txt} }
        child { node {program.cxx} }
        child { node {aux.cxx} }
        child { node {aux.h} }
      }
      child { node {install}
        child { node {program} }
        child { node {libauxlib.a} }
      }
      ;
    \end{tikzpicture}
%%       [dir [src [CMakeLists.txt] [program.cxx] [aux.cxx] [aux.h] ]
%%         [install [program] [libauxlib.a] ] ]
  \end{minipage}
\end{floatingfigure}
%
In order to create a library we use \indexcmake{add_library},
and we link it into the target program with \indexcmake{target_link_libraries}.

By default the library is build as a static \texttt{.a} file,
but adding
\begin{lstlisting}
add_library( auxlib SHARED )
\end{lstlisting}
or adding a runtime flag
\begin{lstlisting}[language=bash]
cmake -D BUILD_SHARED_LIBS=TRUE
\end{lstlisting}
changes that to a shared \texttt{.so} type.

Related: the \indextermtt{-fPIC} compile option
is set by \indexcmake{CMAKE_POSITION_INDEPENDENT_CODE}.

The full \indexterm{CMake} file:
%
\lstinputlisting{tutorials/cmake/withlib/CMakeLists.txt}

Note that we give a path to the library files.
This is interpreted relative to the current directory,
(as of \cmakestandard{3.13});
this current directory is available as \indexcmake{CMAKE_CURRENT_SOURCE_DIR}.

\newpage
\Level 1 {Using subdirectories during the build}

(The files for this examples are in \n{tutorials/cmake/includedir}.)

\begin{tabular}{lp{3in}}
  \toprule
  Commands learned in this section\\
  \midrule
  \lstinline+target_include_directories+&indicate include directories needed\\
  \lstinline+target_sources+&specify more sources for a target\\
  \lstinline+CMAKE_CURRENT_SOURCE_DIR+&variable that expands to the current directory\\
  \lstinline+file+&define single-name synonym for multiple files\\
  \lstinline+GLOB+&define single-name synonym for multiple files\\
  \bottomrule
\end{tabular}

\begin{floatingfigure}[r]{2.5in}
  \begin{minipage}{2.5in}
    \tikzsetnextfilename{cmake-includedir}
    \begin{tikzpicture}[dirtree]
      \node{dir}
      child { node {src}
        child { node {CMakeLists.txt} }
        child { node {program.cxx} }
        child { node {src}
          child { node {aux.cxx} } }
        child { node {inc}
          child { node {aux.h} }
        }
      }
      child { node {install}
        child { node {program} }
      }
      ;
    \end{tikzpicture}
%%       [dir [src [CMakeLists.txt] [program.cxx]
%%           [src [aux.cxx] ]
%%           [inc [aux.h] ]
%%         [install [program] ] ] ]
  \end{minipage}
\end{floatingfigure}
%
Suppose you have a directory with header files,
as in the diagram on the right.
The main program would have
\begingroup\lstset{language=C++}
\lstinputlisting{tutorials/cmake/includedir/program.cxx}
\endgroup
and which is compiled as:
\begin{lstlisting}[language=bash]
cc -c program.cxx -I./inc
\end{lstlisting}
To make sure the header file gets found during the build,
you specify that include path with \indexcmake{target_include_directories}:
\begin{lstlisting}
target_include_directories(
	program PRIVATE
	"${CMAKE_CURRENT_SOURCE_DIR}/inc" )
\end{lstlisting}
It is best to make such paths relative to
\indexcmake{CMAKE_CURRENT_SOURCE_DIR},
or the source root \indexcmake{CMAKE_SOURCE_DIR},
or equivalently \indexcmake{PROJECT_SOURCE_DIR}

\begin{floatingfigure}[r]{2.5in}\vskip3in\end{floatingfigure}
%
Usually, when you start making such directory structure,
you will also have sources in subdirectories.
If you only need to compile them into the main executable,
you could list them into a variable
\begin{lstlisting}
set( SOURCES program.cxx src/aux.cxx )
\end{lstlisting}
and use that variable.
However, this is deprecated practice;
it is recommended to use \indexcmake{target_sources}:
\begin{lstlisting}
target_sources( program PRIVATE src/aux1.cxx src/aux2.cxx )
\end{lstlisting}
Use of a wildcard is not trivial:
\begin{lstlisting}
file( GLOB AUX_FILES "src/*.cxx" )
target_sources( program PRIVATE ${AUX_FILES} )
\end{lstlisting}

Complete \indexterm{CMake} file:
%
\lstinputlisting{tutorials/cmake/includedir/CMakeLists.txt}

\newpage
\Level 1 {Recursive building; rpath}
\label{sec:cmake-install-dirs}
\label{sec:cmake-rpath}

(The files for this examples are in \n{tutorials/cmake/publiclib}.)

\begin{tabular}{lp{3in}}
  \toprule
  Commands learned in this section\\
  \midrule
  \lstinline+add_subdirectory+&declare a subdirectory where cmake needs to be run\\
  \lstinline+CMAKE_CURRENT_SOURCE_DIR+&directory where this command is evaluated\\
  \lstinline+CMAKE_CURRENT_BINARY_DIR+&\\
  \lstinline+LIBRARY_OUTPUT_PATH+&\\
  \lstinline+FILES_MATCHING PATTERN+&wildcard indicator\\
  \bottomrule
\end{tabular}

\begin{floatingfigure}[r]{2.5in}
  \begin{minipage}{2.5in}
    \tikzsetnextfilename{cmake-publiblib}
    \begin{tikzpicture}[dirtree]
      \node{dir}
      child { node {src}
        child { node {CMakeLists.txt} }
        child { node {program.cxx} }
        child { node {lib}
          child { node {CMakeLists.txt} }
          child { node {aux.cxx} }
          child { node {aux.h} }
        }
      }
      child { node {install}
        child { node {program} }
        child { node {lib}
          child { node {libauxlib.so} }
        }
        child { node {include}
          child { node {aux.h} }
        }
      }
      ;
    \end{tikzpicture}
%%       [dir [src [CMakeLists.txt] [program.cxx]
%%           [lib [CMakeLists.txt] [aux.cxx] [aux.h] ]
%%         [install [program] [lib [libauxlib.so] ] ] ] ]
  \end{minipage}
\end{floatingfigure}
%
If your sources are spread over multiple directories,
there needs to be a \texttt{CMakeLists.txt} file in each,
and you need to declare the existence of those directories.
Let's start with the obvious choice of putting library files in a \n{lib} directory
with \indexcmake{add_subdirectory}:
\begin{lstlisting}
add_subdirectory( lib )
\end{lstlisting}
For instance, a library directory would have a \n{CMakeLists.txt} file:
%
\lstinputlisting{tutorials/cmake/publiclib/lib/CMakeLists.txt}
%
to build the library file from the sources indicated,
and to install it in a \n{lib} subdirectory.

We also add a clause to install the header files in an include directory:
\begin{lstlisting}
install( FILES aux.h DESTINATION include )
\end{lstlisting}
For installing multiple files, use
\begin{lstlisting}
install(DIRECTORY ${CMAKE_CURRENT_BINARY_DIR}
    DESTINATION ${LIBRARY_OUTPUT_PATH}
    FILES_MATCHING PATTERN "*.h")
\end{lstlisting}

One problem is to tell the executable where to find the library.
For this we use the
\emph{rpath}\index{rpath!in CMake}
mechanism.
(See section~\ref{sec:ld_rpath}.)
By default, \indexterm{CMake} sets it so that the executable in the build location
can find the library.
If you use a non-trivial install prefix, the following lines work:
\begin{lstlisting}
set( CMAKE_INSTALL_RPATH "${CMAKE_INSTALL_PREFIX}/lib" )
set( CMAKE_INSTALL_RPATH_USE_LINK_PATH TRUE )
\end{lstlisting}
Note that these have to be specified before the target.

The whole file:
%
\lstinputlisting{tutorials/cmake/publiclib/CMakeLists.txt}

\newpage
\Level 1 {Programs that use other libraries}

So far we have discussed executables and libraries that can be used by themselves.
What to if your build result needs external libraries?
We will discuss how to find those libraries in section~\ref{sec:cmake-external};
here we point out their use.

The issue is that these libraries need to be findable
when someone uses your binary. There are two strategies:
\begin{itemize}
\item make sure they have been added to the \indexunix{LD_LIBRARY_PATH};
\item have the \indexterm{linker} add their location through the \indexunix{rpath}
  mechanism to the binary itself.
  This second option is in fact the only one on recent versions of
  \indextermbus{Apple}{Mac OS} because of `System Integrity Protection'.
\end{itemize}

As an example, assume your binary needs the \indexunix{Catch2}
and \indexunix{fmtlib} libraries.
You would then,
in addition to the \indexcmake{target_link_directories} specification,
have:
\begin{lstlisting}
set_target_properties(
	${PROGRAM_NAME} PROPERTIES
	BUILD_RPATH   "${CATCH2_LIBRARY_DIRS};${FMTLIB_LIBRARY_DIRS}"
	INSTALL_RPATH "${CATCH2_LIBRARY_DIRS};${FMTLIB_LIBRARY_DIRS}"
    )
\end{lstlisting}

\Level 1 {Header-only libraries}

Use the \indexcmake{INTERFACE} keyword.

\message{TODO make example of header-only lib}

\Level 0 {Finding and using external packages}
\label{sec:cmake-external}

If your program depends on other libraries, there is a variety of ways
to let \indexterm{CMake} find them.

\Level 1 {CMake commandline options}

(The files for this example are in \n{tutorials/cmake/usepubliclib}.)

You can indicate the location of your external library explicitly on the commandline.

\begin{lstlisting}
cmake -D OTHERLIB_INC_DIR=/some/where/include
      -D OTHERLIB_LIB_DIR=/somewhere/lib
\end{lstlisting}

Example \indexterm{CMake} file:
%
\lstinputlisting{tutorials/cmake/usepubliclib/CMakeLists.txt}

\Level 1 {Package finding through `find library' and `find package'}

%% https://izzys.casa/2020/12/how-to-find-packages-with-cmake-the-basics/

\begin{tabular}{lp{3in}}
  \toprule
  Commands learned in this section\\
  \midrule
  \lstinline+find_library+&find a library with a \n{FOOConfig.cmake} file\\
  \lstinline+CMAKE_PREFIX_PATH+&used for finding \n{FOOConfig.cmake} files\\
  \lstinline+find_package+&find a library with a \n{FindFOO} module\\
  \lstinline+CMAKE_MODULE_PATH+&location for \n{FindFOO} modules\\
  \bottomrule
\end{tabular}

The \indexcmake{find_package} command looks for files with
a name \n{FindXXX.cmake}, which are searched
on the \indexcmake{CMAKE_MODULE_PATH}.
Unfortunately, the working of \indexcmake{find_package}
depend somewhat on the specific package.
For instance, most packages set a variable \lstinline{FooFound}
that you can test
\begin{lstlisting}
find_package( Foo )
if ( FooFound )
    # do something
else()
    # throw an error
endif()
\end{lstlisting}

Some libraries come with a \n{FOOConfig.cmake} file,
which is searched on the \indexcmake{CMAKE_PREFIX_PATH}
through \indexcmake{find_library}.
You typically set this variable to the root of the package installation,
and CMake will find the directory of the \n{.cmake} file.

You can test the variables set:
\begin{lstlisting}
find_library( FOOLIB foo )
if (FOOLIB)
  target_link_libraries( myapp PRIVATE ${FOOLIB} )
else()
  # throw an error
endif()
\end{lstlisting}

\Level 2 {Example: Range-v3}

\begin{lstlisting}
find_package( range-v3 REQUIRED )
target_link_libraries( ${PROGRAM_NAME} PUBLIC range-v3::range-v3 )
\end{lstlisting}

\Level 1 {Use of other packages through `pkg config'}

These days, many package support the \indextermunix{pkgconfig} mechanism.
\begin{enumerate}
\item Suppose you have a library \n{mylib}, installed in \n{/opt/local/mylib}.
\item If \n{mylib} supports \n{pkgconfig}, there is most likely a path
  \n{/opt/local/mylib/lib/pkgconfig}, containing a file \n{mylib.pc}.
\item Add the path that contains the \indextermunix{.pc} file to the
  \indextermunix{PKG_CONFIG_PATH} environment variable.
\end{enumerate}
Cmake is now able to find \n{mylib}:
\begin{lstlisting}
find_package( PkgConfig REQUIRED )
pkg_check_modules( MYLIBRARY REQUIRED mylib )
\end{lstlisting}

This defines variables
\begin{lstlisting}[language=bash]
MYLIBRARY_INCLUDE_DIRS
MYLIBRARY_LIBRARY_DIRS
MYLIBRARY_LIBRARIES
\end{lstlisting}
which you can then use in the
\indexcmake{target_include_directories}
and
\indexcmake{target_link_directories}~\
\indexcmake{target_link_libraries}
commands.

\Level 1 {Writing your own pkg config}

We extend the configuration of section~\ref{sec:cmake-install-dirs}
to generate a \n{.pc} file.

First of all we need a template for the \n{.pc} file:
%
\lstinputlisting{tutorials/cmake/pclib/pclib.pc.in}
%
Here the at-signs delimit CMake macros that will be substituted
when the \n{.pc} file is generated;
Dollar macros go into the file unchanged.

Generating the file is done by the following lines in the CMake configuration:
%
\begin{lstlisting}
set( libtarget auxlib )
configure_file(
    ${CMAKE_CURRENT_SOURCE_DIR}/${PROJECT_NAME}.pc.in
    ${CMAKE_CURRENT_BINARY_DIR}/${PROJECT_NAME}.pc
    @ONLY
)  
install(
    FILES ${CMAKE_CURRENT_BINARY_DIR}/${PROJECT_NAME}.pc
    DESTINATION share/pkgconfig
)
\end{lstlisting}

\Level 1 {Libraries}

\Level 2 {Example: MPI}
\index{MPI!cmake integration|(}

(The files for this example are in \n{tutorials/cmake/mpiprog}.)

While many MPI implementations have a \n{.pc} file,
it's better to use the FindMPI module.
This package defines a number of variables
that can be used to query the MPI found;
for details see \url{https://cmake.org/cmake/help/latest/module/FindMPI.html}
Sometimes it's necessary to set the \indextermtt{MPI_HOME}
environment variable to aid in discovery of the MPI package.

C version:
%
\lstinputlisting{tutorials/cmake/mpiprog/CMakeLists.txt}

Fortran version:
%
\lstinputlisting{tutorials/cmake/mpiprogf/CMakeLists.txt}

The \emph{MPL}\index{MPL!CMake integration} package also uses \indexcmake{find_package}:
\begin{lstlisting}
find_package( mpl REQUIRED )
add_executable( ${PROJECT_NAME} )
target_sources( ${PROJECT_NAME} PRIVATE ${PROJECT_NAME}.cxx )
target_include_directories(
        ${PROJECT_NAME} PUBLIC
        mpl::mpl )
target_link_libraries(
        ${PROJECT_NAME} PUBLIC
        mpl::mpl )
\end{lstlisting}

\index{MPI!cmake integration|)}

\Level 2 {Example: OpenMP}
\label{sec:cmake-omp}
\index{OpenMP!cake integration|(}

\begin{lstlisting}
find_package(OpenMP)
if(OpenMP_C_FOUND) # or CXX
else()
	message( FATAL_ERROR "Could not find OpenMP" )
endif()
# for C:
add_executable( ${program} ${program}.c )
target_link_libraries( ${program} PUBLIC OpenMP::OpenMP_C )
# for C++:
add_executable( ${program} ${program}.cxx )
target_link_libraries( ${program} PUBLIC OpenMP::OpenMP_CXX)
# for Fortran
enable_language(Fortran)
# test: if( OpenMP_Fortran_FOUND )
add_executable( ${program} ${program}.F90 )
target_link_libraries( ${program} PUBLIC OpenMP::OpenMP_Fortran )
\end{lstlisting}

\index{OpenMP!cake integration|)}

\Level 2 {CUDA}

\begin{verbatim}
FindCUDA

Changed in version 3.27: This module is available only if policy CMP0146 is not set to NEW. Port projects to CMake's first-class CUDA language support.

Deprecated since version 3.10: Do not use this module in new code.

It is no longer necessary to use this module or call find_package(CUDA) for compiling CUDA code. Instead, list CUDA among the languages named in the top-level call to the project() command, or call the enable_language() command with CUDA. Then one can add CUDA (.cu) sources directly to targets similar to other languages.

New in version 3.17: To find and use the CUDA toolkit libraries manually, use the FindCUDAToolkit module instead. It works regardless of the CUDA language being enabled.
\end{verbatim}

\Level 2 {Example: MKL}
\index{MKL!cake integration|(}

(The files for this example are in \n{tutorials/cmake/mklcmake}.)

Intel compiler installations come with \indexterm{CMake} support:
there is a file \n{MKLConfig.cmake}.

Example program using \indexterm{Cblas} from MKL:
\lstinputlisting{tutorials/cmake/mklcmake/program.cxx}

The following configuration file lists the various options and such:
%
\lstinputlisting{tutorials/cmake/mklcmake/CMakeLists.txt}

\index{MKL!cake integration|)}

\Level 2 {Example: PETSc}
\index{PETSc!cmake integration|(}

(The files for this example are in \n{tutorials/cmake/petscprog}.)

This \indexterm{CMake} setup searches for \n{petsc.pc},
which is located in \verb+$PETSC_DIR/$PETSC_ARCH/lib/pkgconfig+:
%
\lstinputlisting{tutorials/cmake/petscprog/CMakeLists.txt}

\index{PETSc!cmake integration|)}

\Level 2 {Example: Eigen}
\index{Eigen!cake integration|(}

(The files for this example are in \n{tutorials/cmake/eigen}.)

The \indexterm{eigen} package uses \indextermunix{pkgconfig}.

\lstinputlisting{tutorials/cmake/eigen/CMakeLists.txt}

\index{Eigen!cake integration|)}

\Level 2 {Example: cxxopts}
\index{cxxopts!cmake integration|(}

(The files for this example are in \n{tutorials/cmake/cxxopts}.)

The \indexterm{cxxopts} package uses \indextermunix{pkgconfig}.

\lstinputlisting{tutorials/cmake/cxxopts/CMakeLists.txt}

\index{cxxopts!cmake integration|)}

\Level 2 {Example: fmtlib}
\index{fmtlib!cmke integration|(}

(The files for this example are in \n{tutorials/cmake/fmtlib}.)

In the following example, we use the \indexterm{fmtlib}.
The main \indexterm{CMake} file:
%
\lstinputlisting{tutorials/cmake/fmtlib/CMakeLists.txt}

\Level 2 {Example: fmtlib used in library}

(The files for this example are in \n{tutorials/cmake/fmtliblib}.)

We continue using the \indexterm{fmtlib} library,
but now the generated library also has references to this library,
so we use \lstinline+target_link_directories+ and \lstinline+target_link_library+.

Main file:
%
\lstinputlisting{tutorials/cmake/fmtliblib/CMakeLists.txt}

Library file:
%
\lstinputlisting{tutorials/cmake/fmtliblib/prolib/CMakeLists.txt}

\index{fmtlib!cmake integration|)}

\Level 0 {Customizing the compilation process}

\begin{tabular}{lp{3in}}
  \toprule
  Commands learned in this section\\
  \midrule
  \lstinline+add_compile_options+&global compiler options\\
  \lstinline+target_compile_features+&compiler-independent specification of compile flags\\
  \lstinline+target_compile_definitions+&pre-processor flags\\
  \bottomrule
\end{tabular}

\Level 1 {Customizing the compiler}

It's probably a good idea to tell \indexterm{CMake} explicitly what compiler you are using,
otherwise it may find some default gcc version that came with your system.
Use the variables \indexcmake{CMAKE_CXX_COMPILER}, \indexcmake{CMAKE_C_COMPILER},
\indexcmake{CMAKE_Fortran_COMPILER},
\indexcmake{CMAKE_LINKER}.

Alternatively, set environment variables
\indexunix{CC}, \indexunix{CXX}, \indexunix{FC}
by the explicit paths of the compilers.
For examples, for Intel compilers:
\begin{lstlisting}
export CC=`which icc`
export CXX=`which icpc`
export FC=`which ifort`
\end{lstlisting}

\Level 1 {Global and target flags}

Most of the time, compile options should be associated with a target.
For instance, some file could need a higher or lower optimization level,
or a specific C++ standard.
In that case, use \indexcmake{target_compile_features}.

Certain options may need to be global, in which case you use
\indexcmake{add_compile_options}. Example:
\begin{lstlisting}
## from https://youtu.be/eC9-iRN2b04?t=1548
if (MVSC)
    add_compile_options(/W3 /WX)
else()
    add_compile_options(-W -Wall -Werror)
endif()
\end{lstlisting}

\Level 2 {Universal flags}

Certain flags have a universal meaning, but compiler-dependent realization.
For instance, to specify the C++ standard:
\begin{lstlisting}
target_compile_features( mydemo PRIVATE cxx_std_17 )
\end{lstlisting}
Alteratively, you can set this one the commandline:
\begingroup\lstset{language=Bash}
\begin{lstlisting}
cmake -D CMAKE_CXX_STANDARD=20
\end{lstlisting}
\endgroup
The variable \indexcmake{CMAKE_CXX_COMPILE_FEATURES}
contains the list of all features you can set.

Optimization flags can be set by specifying the \indexcmake{CMAKE_BUILD_TYPE}:
\begin{itemize}
\item 
    \indexcmake{Debug} corresponds to the \n{-g} flag;
\item
    \indexcmake{Release} corresponds to \n{-O3 -DNDEBUG};
\item
    \indexcmake{MinSizeRel} corresponds to \n{-Os -DNDEBUG}
\item 
    \indexcmake{RelWithDebInfo} corresponds to \n{-O2 -g -DNDEBUG}.
\end{itemize}
This variable will often be set from the commandline:
\begin{lstlisting}[language=bash]
cmake .. -DCMAKE_BUILD_TYPE=Release
\end{lstlisting}
Unfortunately, this seems to be the only way to influence optimization flags,
other than explicitly setting compiler flags; see next point.

\Level 2 {Custom compiler flags}

Set the variable
\indexcmake{CMAKE_CXX_FLAGS}
or
\indexcmake{CMAKE_C_FLAGS};
also
\indexcmake{CMAKE_LINKER_FLAGS}
(but see section~\ref{sec:cmake-rpath}
for the popular \indextermtt{rpath} options.)

\Level 1 {Macro definitions}

\indexterm{CMake} can provide macro definitions:
\begin{lstlisting}
target_compile_definitions
  ( programname PUBLIC
    HAVE_HELLO_LIB=1 )
\end{lstlisting}
and your source could test these:
\lstset{language=C}
\begin{lstlisting}
#ifdef HAVE_HELLO_LIB
#include "hello.h"
#endif
\end{lstlisting}
\lstset{language=CMake}

\Level 0 {CMake scripting}
\label{sec:cmake-script}

\begin{tabular}{lp{3in}}
  \toprule
  Commands learned in this section\\
  \midrule
  \lstinline+option+&query a commandline option\\
  \lstinline+message+&trace message during cmake-ing\\
  \lstinline+set+&set the value of a variable\\
  \lstinline+CMAKE_SYSTEM_NAME+&variable containing the operating system name\\
  \lstinline+STREQUALS+&string comparison operator\\
  \bottomrule
\end{tabular}

The \indextermtt{CMakeLists.txt} file is a script,
though it doesn't much look like it.
\begin{itemize}
\item
  Instructions consist of a command, followed by a parenthesized
  list of arguments.
\item
  (All arguments are strings: there are no numbers.)
\item
  Each command needs to start on a new line, but otherwise
  whitespace and line breaks are ignored.
\end{itemize}

Comments start with a hash character.

\Level 1 {System dependencies}

\begin{lstlisting}
if (CMAKE_SYSTEM_NAME STREQUALS "Window")
  target_compile_options( myapp PRIVATE /W4 )
elseif (CMAKE_SYSTEM_NAME STREQUALS "Darwin" -Wall -Wextra -Wpedantic)
  target_compile_options( myapp PRIVATE /W4 )
endif()
\end{lstlisting}

\Level 1 {Messages, errors, and tracing}

The \indexcmake{message} command can be used to write output
to the console. This command has two arguments:
\begin{lstlisting}
message( STATUS "We are rolling!")
\end{lstlisting}
Instead of \indexcmake{STATUS} you can specify other logging levels
(this parameter is actually called `mode' in the documentation);
running for instance
\begin{lstlisting}[language=bash]
cmake --log-level=NOTICE
\end{lstlisting}
will display only messages of `notice' status or higher.

The possibilities here are: \indexcmake{FATAL_ERROR}, \indexcmake{SEND_ERROR},
\indexcmake{WARNING}, \indexcmake{AUTHOR_WARNING}, \indexcmake{DEPRECATION}, \indexcmake{NOTICE},
\indexcmake{STATUS}, \indexcmake{VERBOSE}, \indexcmake{DEBUG}, \indexcmake{TRACE}.

The \indexcmake{NOTICE}, \indexcmake{VERBOSE}, \indexcmake{DEBUG}, \indexcmake{TRACE} options
were added in \cmakestandard{3.15}.

For a complete trace of everything \indexterm{CMake} does, use
the commandline option \n{--trace}.

You can get a verbose make file by using the option
\begin{lstlisting}[language=bash]
-D CMAKE_VERBOSE_MAKEFILE=ON
\end{lstlisting}
on the \indexterm{CMake} invocation. You still need \n{make V=1}.

\Level 1 {Variables}

Variables are set with \texttt{set},
or can be given on the commandline:
\begin{lstlisting}
cmake -D MYVAR=myvalue
\end{lstlisting}
where the space after \n{-D} is optional.

Using the variable by itself gives the value,
except in strings, where a shell-like notation is needed:
\begin{lstlisting}
set(SOME_ERROR "An error has occurred")
message(STATUS "${SOME_ERROR}")
set(MY_VARIABLE "This is a variable")
message(STATUS "Variable MY_VARIABLE has value ${MY_VARIABLE}")
\end{lstlisting}

Variables can also be queried by the \indexterm{CMake} script using the \indexcmake{option} command:
\begin{lstlisting}
option( SOME_FLAG "A flag that has some function" defaultvalue )
\end{lstlisting}

Some variables are set by other commands.
For instance the \texttt{project} command sets
\indexcmake{PROJECT_NAME} and \indexcmake{PROJECT_VERSION}.

\Level 2 {Environment variables}

\emph{Environment variables}\index{environment variable!in Cmake}
can be queried with the \indexcmake{ENV} command:
\begin{lstlisting}
set( MYDIR $ENV{MYDIR} )
\end{lstlisting}

\Level 2 {Numerical variables}

\begin{lstlisting}
math( EXPR lhs_var "math expr" )
\end{lstlisting}

\Level 1 {Control structures}

\Level 2 {Conditionals}

\begin{lstlisting}
if ( MYVAR MATCHES "value$" )
    message( NOTICE "Variable ended in 'value'" )
elseif( stuff )
    message( stuff )
else()
    message( NOTICE "Variable was otherwise" )
endif()
\end{lstlisting}

\Level 2 {Looping}

\begin{lstlisting}
while( myvalue LESS 50 )
    message( stuff )
endwhile()
\end{lstlisting}

\begin{lstlisting}
foreach ( var IN ITEMS item1 item2 item3 )
  ## something wityh ${var}
endforeach()
foreach ( var IN LISTS list1 list2 list3 )
  ## something wityh ${var}
endforeach()
\end{lstlisting}

Integer range, with inclusive bounds, upper bound zero by default:
\begin{lstlisting}
foreach ( idx RANGE 10 )
foreach ( idx RANGE 5 10 )
foreach ( idx RANGE 5 10 2 )
endforeach()
\end{lstlisting}

%% \Level 2 {Things not to do}

%% Do not use macros that affect all targets: \indexcmake{include_directories},
%% \indexcmake{add_definitions}, \indexcmake{link_libraries}.

%% Do not use \indexcmake{target_include_directories} outside your project:
%% that should be found through some of the above mechanisms.


% LocalWords:  Eijkhout CMake workflow CMakeLists txt cmake mylibrary
% LocalWords:  programmain libsource libheader lib hellolibdir

