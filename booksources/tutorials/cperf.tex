% -*- latex -*-
%%%%%%%%%%%%%%%%%%%%%%%%%%%%%%%%%%%%%%%%%%%%%%%%%%%%%%%%%%%%%%%%
%%%%%%%%%%%%%%%%%%%%%%%%%%%%%%%%%%%%%%%%%%%%%%%%%%%%%%%%%%%%%%%%
%%%%
%%%% This text file is part of the source of 
%%%% `The Art of HPC, vol 4: HPC Carpentry'
%%%% by Victor Eijkhout, copyright 2012-6
%%%%
%%%% This book is distributed under a Creative Commons Attribution 3.0
%%%% Unported (CC BY 3.0) license and made possible by funding from
%%%% The Saylor Foundation \url{http://www.saylor.org}.
%%%%
%%%%
%%%%%%%%%%%%%%%%%%%%%%%%%%%%%%%%%%%%%%%%%%%%%%%%%%%%%%%%%%%%%%%%
%%%%%%%%%%%%%%%%%%%%%%%%%%%%%%%%%%%%%%%%%%%%%%%%%%%%%%%%%%%%%%%%

In this tutorial we will discuss some of the techniques that lift you
from being a C~beginner to a `power user'.

\Level 0 {C standards}

The commonly accepted standard for the C~language is
\emph{C99}\index{C!C99}. However, that doesn't mean compilers
automatically accept source using this standard.

Use compiler option:
\begin{verbatim}
cc -std=c99 yourprogram.c
\end{verbatim}
There are some \indextermbus{gcc}{extensions} that require the
\indextermbus{Intel}{compiler} to specify
\begin{verbatim}
icc -std=gnu99 yourprogram.c
\end{verbatim}
instead.

\Level 0 {Allocation}

The easiest way to define an array is to use
\indextermsub{static}{allocation}:
\begin{verbatim}
double array[50];
\end{verbatim}
This has some advantages, such as giving you
\indextermbus{cacheline}{boundary alignment}.

However, since statically allocated arrays are created on the
\indexterm{stack}, they are subject to stacksize limits. Making the
array too large will give a runtime error.

Use the Unix
call
\begin{verbatim}
ulimit -s 123456
\end{verbatim}
to increase the stacksize if needed. 

The alternative is \indextermsub{dynamic}{allocation}, or allocation
on the \indexterm{heap}. This can use all available memory. The
simplest way is by using \indextermtt{malloc}:
\begin{verbatim}
double *array = (double*) malloc( 50*sizeof(double) );
\end{verbatim}
The \indextermtt{sizeof} function is necessary since \n{malloc} takes
an argument in bytes.

The problem with this call is that you lose cacheline alignment, which
can negatively influence performance. A~better option is
\indextermtt{posix_memalign}.

\Level 0 {Compiler defines}

There are various ways of altering program parameters without
recompiling the code. One is to use \indextermbus{compiler}{macros}:
\begin{verbatim}
cc -c -DVALUE=50 myprogram.c
\end{verbatim}
With the \n{-D} option your source acts as if there was a statement
\begin{verbatim}
#define VALUE 50
\end{verbatim}
in it.

To use a default value for the case where you don't specify the
compiler macros, have
\begin{verbatim}
#ifndef VALUE
#define VALUE 23
#endif
\end{verbatim}
in your source.

\Level 0 {Commandline arguments}

\Level 0 {Timers}

See section~\ref{sec:perf-timers}.

% LocalWords:  Eijkhout gcc cacheline stacksize malloc sizeof posix
% LocalWords:  memalign Commandline
