% -*- latex -*-
%%%%%%%%%%%%%%%%%%%%%%%%%%%%%%%%%%%%%%%%%%%%%%%%%%%%%%%%%%%%%%%%
%%%%%%%%%%%%%%%%%%%%%%%%%%%%%%%%%%%%%%%%%%%%%%%%%%%%%%%%%%%%%%%%
%%%%
%%%% This text file is part of the source of 
%%%% `The Art of HPC, vol 4: HPC Carpentry'
%%%% by Victor Eijkhout, copyright 2012-2022
%%%%
%%%% debug.tex : tutorial for debugging
%%%%
%%%%%%%%%%%%%%%%%%%%%%%%%%%%%%%%%%%%%%%%%%%%%%%%%%%%%%%%%%%%%%%%
%%%%%%%%%%%%%%%%%%%%%%%%%%%%%%%%%%%%%%%%%%%%%%%%%%%%%%%%%%%%%%%%

\index{debugging|(}

\begin{quotation}
  \noindent
  Debugging is like being the detective in a crime movie where you are
  also the murderer. (Filipe Fortes, 2013)
\end{quotation}

When a program misbehaves, \emph{debugging} is the process of finding
out \emph{why}.
There are various strategies of finding errors in a program.
The crudest one is debugging by print statements. If you have a
notion of where in your code the error arises, you can edit your code
to insert print statements, recompile, rerun, and see if the output
gives you any suggestions. There are several problems with this:
\begin{itemize}
\item The edit/compile/run cycle is time consuming, especially since
\item often the error will be caused by an earlier section of code,
  requiring you to edit, compile, and rerun repeatedly. Furthermore,
\item the amount of data produced by your program can be too large to
  display and inspect effectively, and
\item if your program is parallel, you probably need to print out data
  from all processors, making the inspection process very tedious.
\end{itemize}

\index{gdb|(}\index{lldb|(}

For these reasons, the best way to debug is by the use of an
interactive \indexterm{debugger}, a program that allows you to monitor
and control the behavior of a running program. In this section you
will familiarize yourself with
\emph{gdb}\index{GNU!gdb|see{gdb}} and
\indexterm{lldb},
the open source
debuggers of the \indexterm{GNU} and \indexterm{clang} projects respectively.
Other debuggers are
proprietary, and typically come with a compiler suite. Another
distinction is that gdb is a commandline debugger; there are
graphical debuggers such as \indexterm{ddd} (a~frontend to gdb) or
\indexterm{DDT} and \indexterm{TotalView} (debuggers for parallel
codes). We limit ourselves to gdb, since it incorporates the basic
concepts common to all debuggers.

In this tutorial you will debug a number of simple programs with
gdb and valgrind. The files can be found in the repository
in the directory \n{code/gdb}.

\begin{table}[p]
  \caption{List of common gdb / lldb commands.}
  
  \begin{tabular}{cc}
    \toprule
    gdb & lldb\\

    \midrule
    \multicolumn{2}{c}{Starting a debugger run} \\
    \midrule
    \n{\$ gdb program}&\n{\$ lldb program}\\
    \n{(gdb) run}&\n{(lldb) run}\\

    \midrule
    \multicolumn{2}{c}{Displaying a stack trace} \\
    \midrule
    \n{(gdb) where}&\n{(lldb) thread backtrace}\\

    \midrule
    \multicolumn{2}{c}{Investigate a specific frame}\\
    \midrule
    \n{frame 2}&\n{frame select 2}\\

    \midrule
    \multicolumn{2}{c}{Run/step}\\
    \midrule
    \n{run / step / continue}&thread continue / step-in/over/out\\

    \midrule
    \multicolumn{2}{c}{Set a breakpoint at a line}\\
    \midrule
    \n{break foo.c:12}&\n{breakpoint set [ -f foo.c ] -l 12}\\
    \n{break foo.c:12 if n>0}&\\
    \n{info breakpoints}&\\

    \midrule
    \multicolumn{2}{c}{Set a breakpoint for exceptions}\\
    \midrule
    \n{catch throw}&\n{break set -E C++}\\
    \bottomrule
  \end{tabular}
\end{table}

\Level 0 {Compiling for debug}

You often need to recompile your code before you can debug it.
A~first reason for this is that the binary code typically knows
nothing about what variable names corresponded to what memory locations,
or what lines in the source to what instructions. In order to make
the binary executable know this, you have to include the \indexterm{symbol table}
in it, which is done by adding the \n{-g} option to the compiler line.

Usually, you also need to lower
the \indextermbus{compiler}{optimization level}: a production code
will often be compiled with flags such as \n{-O2} or \n{-Xhost} that
try to make the code as fast as possible, but for debugging you need
to replace this by~\n{-O0} (`oh-zero').  The reason is that higher
levels will reorganize your code, making it hard to relate the
execution to the source\footnote{Typically, actual code motion is done
by \n{-O3}, but at level \n{-O2} the compiler will inline functions
and make other simplifications.}.

\Level 0 {Invoking the debugger}

There are three ways of using gdb: using it to start a program,
attaching it to an already running program, or using it to inspect a
\indexterm{core dump}. We will only consider the first possibility.

\begin{fntable}{cc}
  \multicolumn{2}{c}{Starting a debugger run} \\
  \midrule
  gdb & lldb\\
  \midrule
  \n{\$ gdb program}&\n{\$ lldb program}\\
  \n{(gdb) run}&\n{(lldb) run}\\
\end{fntable}

Here is an example of how to start gdb with program that has no
arguments (Fortran users, use \n{hello.F}):
\codelisting{tutorials/gdb/c/hello.c}
\begin{verbatim}
%% cc -g -o hello hello.c
# regular invocation:
%% ./hello
hello world
# invocation from gdb:
%% gdb hello
GNU gdb 6.3.50-20050815 # ..... [version info]
Copyright 2004 Free Software Foundation, Inc. .... [copyright info] ....
(gdb) run
Starting program: /home/eijkhout/tutorials/gdb/hello 
Reading symbols for shared libraries +. done
hello world

Program exited normally.
(gdb) quit
%%
\end{verbatim}

Important note: the program was compiled with the \indexterm{debug
  flag}~\n{-g}. This causes the \indexterm{symbol table} (that is, the
translation from machine address to program variables) and other debug
information to be included in the binary. This will make your binary
larger than strictly necessary, but it will also make it slower, for
instance because the compiler will not perform certain
optimizations\footnote{Compiler optimizations are not supposed to
  change the semantics of a program, but sometimes do. This can lead
  to the nightmare scenario where a program crashes or gives incorrect
  results, but magically works correctly with compiled with debug and
  run in a debugger.}.

To illustrate the presence of the symbol table do
\begin{verbatim}
%% cc -g -o hello hello.c
%% gdb hello
GNU gdb 6.3.50-20050815 # ..... version info
(gdb) list
\end{verbatim}
and compare it with leaving out the \n{-g} flag:
\begin{verbatim}
%% cc -o hello hello.c
%% gdb hello
GNU gdb 6.3.50-20050815 # ..... version info
(gdb) list
\end{verbatim}

For a program with commandline input we give the arguments to the
\n{run} command (Fortran users use \n{say.F}):
\begin{multicols}{2}
  \codelisting{tutorials/gdb/c/say.c}
  \vfill\columnbreak
\begin{verbatim}
%% cc -o say -g say.c
%% ./say 2
hello world
hello world
%% gdb say
.... the usual messages ...
(gdb) run 2
Starting program: /home/eijkhout/tutorials/gdb/c/say 2
Reading symbols for shared libraries +. done
hello world
hello world

Program exited normally.
\end{verbatim}
\end{multicols}

\Level 0 {Finding errors: where, frame, print}

Let us now consider some programs with errors.

\Level 1 {C programs}

The following code has several errors.
We will use the debugger to uncover them.

\cverbatimsnippet{gdb-square}
\begin{verbatim}
%% cc -g -o square square.c
%% ./square
5000
Segmentation fault
\end{verbatim}
The \indexterm{segmentation fault} (other messages are possible too) 
indicates that we are accessing
memory that we are not allowed to, making the program exit.
A~debugger will quickly tell us where this happens:
\begin{verbatim}
%% gdb square
(gdb) run
50000

Program received signal EXC_BAD_ACCESS, Could not access memory.
Reason: KERN_INVALID_ADDRESS at address: 0x000000000000eb4a
0x00007fff824295ca in __svfscanf_l ()
\end{verbatim}

Apparently the error occurred in a function \n{__svfscanf_l}, which is
not one of ours, but a system function. Using the \n{backtrace}
(or~\n{bt}, also \n{where} or~\n{w}) command we
display the \indexterm{call stack}. This usually allows us to find out
where the error lies:

\begin{fntable}{cc}
  \multicolumn{2}{c}{Displaying a stack trace} \\
  \midrule
  gdb & lldb\\
  \midrule
  \n{(gdb) where}&\n{(lldb) thread backtrace}\\
\end{fntable}

{\small
\begin{verbatim}
(gdb) where
#0  0x00007fff824295ca in __svfscanf_l ()
#1  0x00007fff8244011b in fscanf ()
#2  0x0000000100000e89 in main (argc=1, argv=0x7fff5fbfc7c0) at square.c:7
\end{verbatim}
}

We inspect the actual problem:

  \begin{fntable}{ll}
    \multicolumn{2}{c}{Investigate a specific frame}\\
    \midrule
    gdb&clang\\
    \n{frame 2}&\n{frame select 2}\\
  \end{fntable}

We take a close look at line~7, and see that we need to
change \n{nmax} to~\n{&nmax}.

There is still an error in our program:
{\small
\begin{verbatim}
(gdb) run
50000

Program received signal EXC_BAD_ACCESS, Could not access memory.
Reason: KERN_PROTECTION_FAILURE at address: 0x000000010000f000
0x0000000100000ebe in main (argc=2, argv=0x7fff5fbfc7a8) at square1.c:9
9           squares[i] = 1./(i*i); sum += squares[i];
\end{verbatim}
}
We investigate further:
\begin{verbatim}
(gdb) print i
$1 = 11237
(gdb) print squares[i]
Cannot access memory at address 0x10000f000
(gdb) print squares
$2 = (float *) 0x0
\end{verbatim}
and we quickly see that we forgot to allocate \n{squares}.

By the way, we were lucky here: this sort of memory errors is not always
detected. Starting our programm with a smaller input does not lead to
an error:
\begin{verbatim}
(gdb) run
50
Sum: 1.625133e+00

Program exited normally.
\end{verbatim}

Memory errors can also occur if we have a legitimate array, but we access it
outside its bounds.
The following program fills an array, forward, and reads it out, backward.
However, there is an indexing error in the second loop.
\cverbatimsnippet{gdb-up}
\begin{verbatim}
Program received signal EXC_BAD_ACCESS, Could not access memory.
Reason: KERN_INVALID_ADDRESS at address: 0x0000000100200000
0x0000000100000f43 in main (argc=1, argv=0x7fff5fbfe2c0) at up.c:15
15          s += array[i];
(gdb) print array
$1 = (double *) 0x100104d00
(gdb) print i
$2 = 128608
\end{verbatim}
You see that the index where the debugger finally complains
is quite a bit larger than the size of the array.
\begin{exercise}
  Can you think of a reason why indexing out of bounds is not immediately fatal?
  What would determine where it does become a problem?
  (Hint: how is computer memory structured?)
\end{exercise}

In section~\ref{sec:valgrind} you will see a tool
that spots any out-of-bound indexing.

\Level 1 {Fortran programs}

Compile and run the following program:
%%\codelisting{tutorials/gdb/f/square.F}
\fverbatimsnippet{gdb-squaref}
It should end prematurely with a message such as `Illegal instruction'.
Running the program in gdb quickly tells you where the problem lies:
\begin{verbatim}
(gdb) run
Starting program: tutorials/gdb//fsquare 
Reading symbols for shared libraries ++++. done

Program received signal EXC_BAD_INSTRUCTION,
Illegal instruction/operand.
0x0000000100000da3 in square () at square.F:7
7                sum = sum + squares(i)
\end{verbatim}
We take a close look at the code and see that we did not allocate
\n{squares} properly.

\Level 0 {Stepping through a program}

\begin{fntable}{lll}
  \multicolumn{3}{c}{Stepping through a program}\\
  \midrule
  gdb&lldb&meaning\\
  \n{run}&&start a run\\
  \n{cont}&&continue from breakpoint\\
  \n{next}&&next statement on same level\\
  \n{step}&&next statement, this level or next\\
\end{fntable}

Often the error in a program is sufficiently obscure that you need to
investigate the program run in detail. Compile the following program
%
%%\codelisting{tutorials/gdb/c/roots.c}
\cverbatimsnippet{gdbrootsc}
%
and run it:
\begin{verbatim}
%% ./roots
sum: nan
\end{verbatim}
Start it in gdb as before:
\begin{verbatim}
%% gdb roots
GNU gdb 6.3.50-20050815 
Copyright 2004 Free Software Foundation, Inc.
....
\end{verbatim}
but before you run the program, you set a \indexterm{breakpoint}
at \n{main}.
This tells the execution to stop, or `break', in the main program.
\begin{verbatim}
(gdb) break main
Breakpoint 1 at 0x100000ea6: file root.c, line 14.
\end{verbatim}
Now the program will stop at the first executable statement in \n{main}:
\begin{verbatim}
(gdb) run
Starting program: tutorials/gdb/c/roots
Reading symbols for shared libraries +. done

Breakpoint 1, main () at roots.c:14
14        float x=0;
\end{verbatim}

Most of the time you will set a breakpoint at a specific line:

\begin{fntable}{ll}
  \multicolumn{2}{c}{Set a breakpoint at a line}\\
  \midrule
  gdb&lldb\\
  \n{break foo.c:12}&\n{breakpoint set [ -f foo.c ] -l 12}\\
\end{fntable}

If execution is stopped at a breakpoint, you can do various things,
such as issuing the \n{step} command:
\begin{verbatim}
Breakpoint 1, main () at roots.c:14
14        float x=0;
(gdb) step
15        for (i=100; i>-100; i--)
(gdb) 
16          x += root(i);
(gdb) 
\end{verbatim}
(if you just hit return, the previously issued command is
repeated). Do a number of \n{step}s in a row by hitting return. What
do you notice about the function and the loop?

Switch from doing \n{step} to doing \n{next}. Now what do you notice
about the loop and the function? 

Set another breakpoint: \n{break 17} and do \n{cont}. What happens?

Rerun the program after you set a breakpoint on the line with the
\n{sqrt} call. When the execution stops there do \n{where} and
\n{list}.

\begin{itemize}
\item If you set many breakpoints, you can find out what they are with
  \n{info breakpoints}. 
\item You can remove breakpoints with \n{delete n} where \n{n} is the
  number of the breakpoint.
\item If you restart your program with \n{run} without leaving gdb,
  the breakpoints stay in effect.
\item If you leave gdb, the breakpoints are cleared but you can save
  them: \n{save breakpoints <file>}. Use \n{source <file>} to read
  them in on the next gdb run.
\end{itemize}

\Level 0 {Inspecting values}

Run the previous program again in gdb: set a breakpoint at the line
that does the \n{sqrt} call before you actually call \n{run}. When the
program gets to line~8 you can do \n{print n}. Do \n{cont}. Where does
the program stop?

If you want to repair a variable, you can do \n{set var=value}. Change
the variable \n{n} and confirm that the square root of the new value
is computed. Which commands do you do?

\Level 0 {Breakpoints}
\index{breakpoint|(textbf}

If a problem occurs in a loop, it can be tedious keep typing \n{cont}
and inspecting the variable with \n{print}. Instead you can add a
condition to an existing breakpoint. First of all, you can make the breakpoint
subject to a condition: with
\begin{verbatim}
condition 1 if (n<0)
\end{verbatim}
breakpoint~1 will only obeyed if \texttt{n<0} is true.

You can also have a breakpoint that is only activated by some condition.
The statement
\begin{verbatim}
break 8 if (n<0)
\end{verbatim}
means that breakpoint~8 becomes (unconditionally) active after
the condition \texttt{n<0} is encountered.

\begin{fntable}{cc}
  \multicolumn{2}{c}{Set a breakpoint} \\
  \midrule
  gdb & lldb\\
  \midrule
    \n{break foo.c:12}&\n{breakpoint set [ -f foo.c ] -l 12}\\
    \n{break foo.c:12 if n>0}&\\
\end{fntable}


\begin{remark}
  You can break on \n{NaN} with the following trick:
\begin{verbatim}
break foo.c:12 if x!=x
\end{verbatim}
  using the fact that \n{NaN} is the only number not equal to itself.
\end{remark}

Another possibility is to use \n{ignore 1 50}, which will not stop at
breakpoint 1 the next 50 times.

Remove the existing breakpoint, redefine it with the condition \n{n<0}
and rerun your program. When the program breaks, find for what value
of the loop variable it happened. What is the sequence of commands you use?

You can set a breakpoint in various ways:
\begin{itemize}
\item \n{break foo.c} to stop when code in a certain file is reached;
\item \n{break 123} to stop at a certain line in the current file;
\item \n{break foo} to stop at subprogram \n{foo}
\item or various combinations, such as \n{break foo.c:123}.
\end{itemize}

Information about breakpoints:

\begin{itemize}
\item If you set many breakpoints, you can find out what they are with
  \n{info breakpoints}. 
\item You can remove breakpoints with \n{delete n} where \n{n} is the
  number of the breakpoint.
\item If you restart your program with \n{run} without leaving gdb,
  the breakpoints stay in effect.
\item If you leave gdb, the breakpoints are cleared but you can save
  them: \n{save breakpoints <file>}. Use \n{source <file>} to read
  them in on the next gdb run.
\item In languages with \emph{exceptions}, such
  as~\emph{C++}\index{C++!exception}, you can set a \indexterm{catchpoint}:

  \begin{fntable}{ll}
    \multicolumn{2}{c}{Set a breakpoint for exceptions}\\
    \midrule
    gdb&clang\\   \n{catch throw}&\n{break set -E C++}\\
  \end{fntable}

\end{itemize}

Finally, you can execute commands at a breakpoint:
\begin{verbatim}
break 45
command
print x
cont
end
\end{verbatim}
This states that at line 45 variable~\n{x} is to be printed, and execution
should immediately continue.

If you want to run repeated gdb sessions on the same program,
you may want to save an reload breakpoints. This can be done with
\begin{verbatim}
save-breakpoint filename
source filename
\end{verbatim}

\index{breakpoint|)}
\index{gdb|)}

\Level 0 {Memory debugging}

Many problems in programming stem from memory errors. We start with a
sort description of the most common types, and then discuss tools that
help you detect them.

\Level 1 {Type of memory errors}

\Level 2 {Invalid pointers}

Dereferencing a pointer that does not point to an allocated object can lead to an error.
If your pointer points into valid memory anyway, your computation will continue but with incorrect results.

However, it is more likely that
your program will probably exit with a
\indexterm{segmentation violation} or a \indexterm{bus error}.

\Level 2 {Out-of-bounds errors}

Addressing outside the bounds of an allocated object is less likely to crash your program and more likely to give incorrect results.

Exceeding bounds by a large enough amount will again give a
segmentation violation, but going out of bounds by a small amount may
read invalid data, or corrupt data of other variables, giving
incorrect results that may go undetected for a long time.

\Level 2 {Memory leaks}

We speak of a \indexterm{memory leak} if allocated memory becomes unreachable.
Example:
\begin{lstlisting}
if (something) {
  double *x = malloc(10*sizeofdouble);
  // do something with x
}
\end{lstlisting}
After the conditional, the allocated memory is not freed, but the
pointer that pointed to has gone away.

This last type especially can be hard to find.
Memory leaks will only surface in that your program runs out of
  memory. That in turn is detectable because your allocation will
  fail. It is a good idea to always check the return result of your
  \indextermtt{malloc} or \indextermtt{allocate} statement!

\Level 0 {Memory debugging with Valgrind}
%%packtsnippet valgrind
\label{sec:valgrind}

Errors leading to memory problems are easy to make.
In this section we will see how \indexterm{valgrind}
makes it possible to track down these errors.
The use of valgrind is simplicity itself:
\begin{lstlisting}
valgrind yourprogram yourargs
\end{lstlisting}

As a first example, consider out of bound addressing,
also known as \indextermbus{buffer}{overflow}:
%
\cxxverbatimsnippet{corruptbound}

This is unlikely to crash your code,
but the results are unpredictable,
and this is certainly a failure of your program logic.

Valgrind indicates that this is an invalid read,
what line it occurs on,
and where the block was allocated:
%
\lstinputlisting{code/corruption/cxx/bound.runout}

\begin{remark}
  Buffer overflows are a well-known security risk,
  typically associated with reading string input
  from a user source.
  Buffer overflows can be largely avoided by using
  C++ constructs such as \lstinline{cin} and \lstinline{string}
  instead of \lstinline{sscanf} and character arrays.
\end{remark}

Valgrind is informative but cryptic, since it works on the bare
memory, not on variables. Thus, these error messages take some
exegesis. They state that line 10 reads a 4-byte object immediately
after a block of 40 bytes that was allocated. In other words: the code
is writing outside the bounds of an allocated array.

The next example performs a read on an array that has already been free'd.
In this simple case you will actually get the expected output,
but if the read comes much later than the free, the output can be anything.
%
\cxxverbatimsnippet{corruptfree}

Valgrind again states that this is an invalid read;
it gives both where the block was allocated and where it was freed.
%
\lstinputlisting{code/corruption/cxx/free.runout}

On the other hand, if you forget to free memory you have a
\indextermbus{memory}{leak}
(just imagine allocation, and not free'ing, in a loop)
%
\cxxverbatimsnippet{corruptleak}

which valgrind reports on:
%
\lstinputlisting{code/corruption/cxx/leak.runout}

Memory leaks are much more rare in C++ than in~C
because of containers such as \lstinline{std::vector}.
However, in sophisticated cases you may still do your own
memory management, and you need to be aware of the danger of memory leaks.

If you do your own memory management, there is also a danger
of writing to an array pointer that has not been allocated yet:
%
\cxxverbatimsnippet{corruptinit}

The behavior of this code depends on all sorts of things:
if the pointer variable is zero, the code will crash.
On the other hand, if it contains some random value,
the write may succeed; provided you are not writing
too far from that location.

The output here shows both the valgrind diagnosis,
and the \ac{OS} message when the program aborted:
%
\lstinputlisting{code/corruption/cxx/leak.runout}

%%packtsnippet end

\Level 1 {Electric fence}

The \indexterm{electric fence} library is one of a number of tools
that supplies a new \indextermtt{malloc} with debugging support.
These are linked instead of the \n{malloc} of the standard
\indextermtt{libc}.

\begin{verbatim}
cc -o program program.c -L/location/of/efence -lefence
\end{verbatim}

Suppose your program has an out-of-bounds error. Running with gdb,
this error may only become apparent if the bounds are exceeded by a
large amount. On the other hand, if the code is linked with
\indextermtt{libefence}, the debugger will stop at the very first time
the bounds are exceeded.

\Level 0 {Further reading}

A good tutorial: \url{http://www.dirac.org/linux/gdb/}.

Reference manual: \url{http://www.ofb.net/gnu/gdb/gdb_toc.html}.

\index{debugging|)}

% LocalWords:  Eijkhout tex Filipe gdb lldb commandline ddd frontend
% LocalWords:  DDT TotalView valgrind backtrace foo svfscanf bt nmax
% LocalWords:  cont var textbf NaN catchpoint malloc sizeofdouble cc
% LocalWords:  printf libc libefence MPI init Allinea ntids mytid
