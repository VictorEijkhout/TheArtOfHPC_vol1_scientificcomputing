%%%%%%%%%%%%%%%%%%%%%%%%%%%%%%%%%%%%%%%%%%%%%%%%%%%%%%%%%%%%%%%%
%%%%%%%%%%%%%%%%%%%%%%%%%%%%%%%%%%%%%%%%%%%%%%%%%%%%%%%%%%%%%%%%
%%%%
%%%% This text file is part of the source of 
%%%% `Introduction to High-Performance Scientific Computing'
%%%% by Victor Eijkhout, copyright 2012/3/4/5
%%%%
%%%% This book is distributed under a Creative Commons Attribution 3.0
%%%% Unported (CC BY 3.0) license and made possible by funding from
%%%% The Saylor Foundation \url{http://www.saylor.org}.
%%%%
%%%%
%%%%%%%%%%%%%%%%%%%%%%%%%%%%%%%%%%%%%%%%%%%%%%%%%%%%%%%%%%%%%%%%
%%%%%%%%%%%%%%%%%%%%%%%%%%%%%%%%%%%%%%%%%%%%%%%%%%%%%%%%%%%%%%%%

A good text editor is an indispensable tool for any
programmer\footnote {Alternatively, you could use an \acf{IDE} such as
\indexterm{Visual Studio} or \indexterm{Eclipse}, but they are
usually harder to customize, not installed on every system, et
cetera. Really: make sure that you learn at least one common text
editor.}. In this tutorial you will learn the basics of the two most
common Unix editors: \indexterm{vi} and \indexterm{emacs}.

\Level 0 {Vi}
\index{vi|(textbf}

\begin{purpose}
  In this section you will learn the basics of text editing with `vi'.
\end{purpose}

Good set of
tips: \url{https://www.cs.oberlin.edu/~kuperman/help/vim/home.html}

The vi editor (pronounced `vee-eye', not `vai') has a `modal' setup:
you are either in input mode, where what you type becomes part of your
file, or you are in edit mode, where your typing is interpreted as
commands.

\Level 1 {Indentation}

Decent settings for new files:
\begin{verbatim}
set smartindent
set shiftwidth=2
syntax on
\end{verbatim}

The command \n{[[=]]} applies the right indentation to the block
surrounding the cursor position.

\index{vi|)}

\Level 0 {Emacs}

\begin{purpose}
  In this section you will learn the basics of text editing with `emacs'.
\end{purpose}

The emacs is always in input mode: ordinary characters you type become
part of the file you are currently editing. To execute commands you
need the `Control' and `Escape' (for historical reasons often called
`Meta') keys.

\Level 1 {Indentation}

Emacs can usually detect the proper indentation scheme from the
name or extension of your file, and it will go into the appropriate
mode automatically. You can also force the mode explicitly by \n{Esc x
  latex-mode} and similar commands.

However, a file does not remember its mode, so if emacs can not deduce
the mode of your file, put a line
\begin{verbatim}
// -*- c++ -*-
\end{verbatim}
at the top of your file. The first non-blank character(s) of this line
are chosen to make it a comment in the language of a file. For
example, to indicate that a \TeX\ file is for \LaTeX, you would use
\begin{verbatim}
% -*- latex -*-
\end{verbatim}

% LocalWords:  Eijkhout IDE textbf vee vai Esc
