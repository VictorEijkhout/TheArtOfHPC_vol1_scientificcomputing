% -*- latex -*-
%%%%%%%%%%%%%%%%%%%%%%%%%%%%%%%%%%%%%%%%%%%%%%%%%%%%%%%%%%%%%%%%
%%%%%%%%%%%%%%%%%%%%%%%%%%%%%%%%%%%%%%%%%%%%%%%%%%%%%%%%%%%%%%%%
%%%%
%%%% This text file is part of the source of 
%%%% `Introduction to High-Performance Scientific Computing'
%%%% by Victor Eijkhout, copyright 2012/3/4/5
%%%%
%%%% This book is distributed under a Creative Commons Attribution 3.0
%%%% Unported (CC BY 3.0) license and made possible by funding from
%%%% The Saylor Foundation \url{http://www.saylor.org}.
%%%%
%%%% bit.tex: tutorial on bitwise operations
%%%%
%%%%%%%%%%%%%%%%%%%%%%%%%%%%%%%%%%%%%%%%%%%%%%%%%%%%%%%%%%%%%%%%
%%%%%%%%%%%%%%%%%%%%%%%%%%%%%%%%%%%%%%%%%%%%%%%%%%%%%%%%%%%%%%%%


In most of this book we consider numbers, such as integer or floating
point representations of real numbers, as our lowest building
blocks. Sometimes, however, it is necessary to dig deeper and consider
the actual representation of such numbers in terms of bits.

Various programming languages have support for bit operations.
We will explore the various options.
For details on C++ and Fortran, see \ISPref{sec:bitc} and \ISPref{sec:bitf} respectively.

\Level 0 {Construction and display}

\Level 1 {C/C++}

The built-in possibilities for display are limited:
\begin{lstlisting}
printf("Octal: %o",i);
printf("Hex  : %x",i);
\end{lstlisting}
gives octal and hexadecimal representation, but there is no
\indexterm{format specifier}
for binary. Instead use the following bit of magic:
\begin{lstlisting}
void printBits(size_t const size, void const * const ptr)
{
  unsigned char *b = (unsigned char*) ptr;
  unsigned char byte;
  for (int i=size-1; i>=0; i--) {
      for (int j=7; j>=0; j--) {
	  byte = (b[i] >> j) & 1;
	  printf("%u", byte);
	}
    }
}
/* ... */
printBits(sizeof(i),&i);  
\end{lstlisting}

\Level 1 {Python}

\begin{itemize}
\item The python \pyinline{int} function converts a string to
  int. A~second argument can indicate what base the string is to be
  interpreted in:
\begin{lstlisting}
five     = int('101',2)
maxint32 = int('0xffffffff',16)
\end{lstlisting}
\item Function \pyinline{bin} \pyinline{hex}
  convert an int to a string in base 2,8,16 respectively.
\item Since python integers can be of unlimited length, there is a
  function to determine the bit length (Python version~3.1):
  \pyinline{i.bit_length()}.
\end{itemize}

\Level 0 {Bit operations}

Boolean operations are usually applied to the boolean datatype of the
programming language. Some languages allow you to apply them to actual
bits.

\begin{tabular}{rlll}
  \toprule
  &boolean&bitwise (C)&bitwise (Py)\\
  and&\verb+&&+ & \verb+&+ & \verb+&+ \\
  or &\verb+||+ & \verb+|+ & \verb+|+ \\
  not&\verb+!+  &          & \verb+~+ \\
  xor&          & \verb+^+ &          \\
  \bottomrule
\end{tabular}

Additionally, there are operations on the bit string as such:

\begin{tabular}{rl}
  \toprule
  left  shift& \verb+<<+ \\
  right shift& \verb+>>+ \\
  \bottomrule
\end{tabular}

\begin{exercise}
  \label{tutex:bit-even}
  Use bit operations to test whether a number is odd or even.
\end{exercise}

The shift operations are sometimes used as an efficient shorthand for
arithmetic. For instance, a left shift by one position corresponds to
a multiplication by two.

\begin{exercise}
  \label{ex:malloc_align}
  Given an integer~$n$, find the largest multiple of~8 that is~$\leq
  n$.

  This mechanism is sometimes used to allocate aligned memory.
  Write a routine
\begin{verbatim}
aligned_malloc( int Nbytes, int aligned_bits );
\end{verbatim}
  that allocates \n{Nbytes} of memory, where the first byte has an
  address that is a multiple of \n{aligned_bits}.
\end{exercise}

% LocalWords:  Eijkhout
