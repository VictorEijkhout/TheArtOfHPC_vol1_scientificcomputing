In this project you will build a cache simulator and analyze the cache
hit/miss behaviour of code, either real or simulated.

\Level 0 {Cache simulation}

A simulated cache is a simple data structure that records for each
cache address what memory address it contains, and how long the data
has been present. Design this data structure and write the access
routines. (Use of an object oriented language is recommended.)

Write your code so that your cache can have various levels of
associativity, and different replacement policies.

For simplicity, do not distinguish between read and write
access. Therefore, the execution of a program becomes a stream of
\begin{verbatim}
cache.access_address( 123456 );
cache.access_address( 70543 );
cache.access_address( 12338383 );
.....
\end{verbatim}
calls where the argument is the memory address. Your code will record
whether the request can be satisfied from cache, or whether the data
needs to be loaded from memory.

\Level 0 {Code simulation}

Find some example codes, for instance from a scientific project you
are involved in, and translate the code to a sequence of memory
address. 

You can also simulate codes by generating a stream of access
instructions as above:
\begin{itemize}
\item Some access will be to random locations, corresponding to use of
  scalar variables;
\item At other times access will be to regularly space addresses,
  corresponding to the use of an array;
\item Array operations with indirect addressing will cause prolonged
  episodes of random address access.
\end{itemize}
Try out various mixes of the above instruction types. For the array
operations, try smaller and larger arrays with various degrees of
reuse.

Does your simulated code behave like a real code?

\Level 0 {Investigation}

First implement a single cache level and investigate the behaviour of
cache hits and misses. Explore different associativity amounts and
different replacement policies.

\Level 0 {Analysis}

Do a statistical analysis of the cache hit/miss behaviour. You can
start with~\cite{Rao:1978:cache}\footnote{Strictly speaking that paper
  is about page swapping out of virtual memory
  (section~\ref{sec:tlb}), but everything translates to cacheline
  swapping out of main memory.}. Hartstein~\cite{Hartstein:cache-sqrt}
found a power law behaviour. Are you finding the same?

