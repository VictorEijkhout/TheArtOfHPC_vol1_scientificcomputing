%%%%%%%%%%%%%%%%%%%%%%%%%%%%%%%%%%%%%%%%%%%%%%%%%%%%%%%%%%%%%%%%
%%%%%%%%%%%%%%%%%%%%%%%%%%%%%%%%%%%%%%%%%%%%%%%%%%%%%%%%%%%%%%%%
%%%%
%%%% This text file is part of the source of 
%%%% `Introduction to High-Performance Scientific Computing'
%%%% by Victor Eijkhout, copyright 2012-2022
%%%%
%%%% This book is distributed under a Creative Commons Attribution 3.0
%%%% Unported (CC BY 3.0) license and made possible by funding from
%%%% The Saylor Foundation \url{http://www.saylor.org}.
%%%%
%%%%
%%%%%%%%%%%%%%%%%%%%%%%%%%%%%%%%%%%%%%%%%%%%%%%%%%%%%%%%%%%%%%%%
%%%%%%%%%%%%%%%%%%%%%%%%%%%%%%%%%%%%%%%%%%%%%%%%%%%%%%%%%%%%%%%%

In this project you will combine some of the theory and practical
skills you learned in class to solve a real world problem. In addition
to writing code that solves the program, you will use the following
software practices:
\begin{itemize}
  \item use source code control,
  \item give the program checkpoint/restart capability, and
  \item use basic profiling and visualization of the results.
\end{itemize}

The Heat Equation (see section~\ref{sec:heateq}) is given by
\[ \frac{\partial T(x,t)}{\partial t}=
\begin{cases}
  \alpha \frac{\partial^2 T(x,y)}{\partial x^2}+q(x,t)&\hbox{1D}\\
  \alpha \frac{\partial^2 T(x,y)}{\partial x^2}+
    \alpha \frac{\partial^2 T(x,y)}{\partial y^2}+q(x,t)&\hbox{2D}\\
    \ldots&\hbox{3D}
\end{cases}
\]
where $t\geq 0$ and $x\in[0,1]$, subject to boundary conditions
\[ T(x,0)=T_0(x),\quad\hbox{for $x\in[0,1]$}, \]
and similar in higher dimensions, and
\[ T(0,t)=T_a(t),\, T(1,t)=T_b(t),\quad\hbox{for $t>0$}. \]
You will solve this problem using the explicit and implicit Euler
methods.

\Level 0 {Software}

As \acp{PDE} go, this is a fairly simple one. In particular the
regular structure makes it easy to code this project using regular
arrays to store the data.
However, you can also
write your software using the PETSc library.
In that case, use the \n{MatMult} routine
for matrix-vector multiplication and \n{KSPSolve} for linear system
solution. Exception: code the Euler methods yourself.

Be sure to use a Makefile for building your project (tutorial~\CARPref{tut:gnumake}).

Add your source files, Makefile, and job scripts to a git repository
(tutorial~\CARPref{tut:git}); do not add binaries or output files. Make
sure that there is a README
file with instructions on how to build and run your code.

Implement a checkpoint/restart facility by writing vector data, size
of the time step, and other necessary items, to an \n{hdf5} file
(tutorial~\CARPref{tut:hdf5}). Your program should be able to read this
file and resume execution.

\Level 0 {Tests}

Do the following tests on a single core.

\subsection*{Method stability}

Run your program for the 1D case with 
\[
\begin{cases}
  q=\sin\ell\pi x\\ T_0(x)=e^x\\ T_a(t)=T_b(t)=0\\ \alpha=1
\end{cases}
\]
Take a space discretization at least $h=10^{-2}$ but especially in
parallel do not be afraid to try large problem sizes. Try various 
time steps
and show that
the explicit method can diverge. What is the maximum time step for
which it is stable? 

For the implicit method, at first use a direct method to solve the
system.
This corresponds to PETSc options \n{KSPPREONLY} and \n{PCLU} (see
section~\ref{sec:cg}).

Now use an iterative method (for instance \n{KSPCG} and \n{PCJACOBI});
is the method still stable? Explore using a low convergence tolerance
and large time steps. 

Since the forcing function~$q$ and the boundary conditions have no
time dependence, the solution $u(\cdot,t)$ will converge to a
\indexterm{steady state} solution~$u_\infty(x)$
as~$t\rightarrow\nobreak\infty$. What is the influence of the time step on the
speed with which implicit method converges to this steady state?

Hint: the steady state is described by $u_t\equiv0$. Substitute this
in the \ac{PDE}. Can you find explicitly what the steady state is?

Run these tests with various values for~$\ell$.

\subsection*{Timing}

If you run your code with the commandline option \n{-log_summary}, you
will get a table of timings of the various PETSc routines. Use that to do
the following timing experiments. Make sure you use a version of PETSc
that was \emph{not} compiled with debug mode.

Construct your coefficient matrix as a dense matrix, rather than
sparse. Report on the difference in total memory, and the runtime and
flop rate of doing one time step. Do this for both the explicit and
implicit method and explain the results.

With a sparse coefficient matrix, report on the timing of a
  single time step. Discuss the respective flop counts and the
  resulting performance.


\subsection*{Restart}

Implement a restart facility: every 10 iterations write out the
values of the current iterate, together with values of $\Delta x$,
$\Delta t$, and~$\ell$. Add a flag \n{-restart} to your program that
causes it to read the restart file and resume execution, reading all
parameters from the restart file.

Run your program for 25 iterations, and restart, causing it to run
again from iteration~20. Check that the values in iterations
$20\ldots25$ match.

\Level 0 {Parallelism}

Do the following tests to determine the parallel scaling of your code.

At first test the explicit method, which should be perfectly
  parallel. Report on actual speedup attained. Try larger and smaller
  problem sizes and report on the influence of the problem size.

The above settings for the implicit method (\n{KSPPREONLY} and
\n{PCLU}) lead to a runtime error. One way out is to let the system be
solved by an iterative method. Read the PETSc manual and web pages to
find out some choices of iterative method and preconditioner and try
them. Report on their efficacy.

\Level 0 {Comparison of solvers}

In the implicit timestepping method you need to solve linear
systems. In the 2D (or 3D) case it can make a big difference whether
you use a direct solver or an iterative one.

Set up your code to run with $q\equiv0$ and zero boundary
  conditions everywhere; start with a nonzero initial solution. You
  should now get convergence to a zero steady state, thus the norm of
  the current solution is the norm of the iterate. 

Now do the following comparison; take several values for the time
step.

\subsection*{Direct solver}

If your PETSc installation includes direct solvers such as
\indexterm{MUMPS}, you can invoke them with
\begin{verbatim}
myprog -pc_type lu -ksp_type preonly \
    -pc_factor_mat_solver_package mumps 
\end{verbatim}
Run your code with a direct solver, both sequentially and in parallel,
and record how long it takes for the error to get down to $10^{-6}$.

\subsection*{Iterative Solver}

Use an iterative solver, for instance \n{KSPCG} and \n{KSPBCGS}.
Experiment with the convergence tolerance: how many timesteps does it
take to get a $10^{-6}$ error if you set the iterative method
tolerance to $10^{-12}$, how much if you take a lesser tolerance?

Compare timings between direct and iterative method.

\Level 0 {Reporting}

Write your report using \LaTeX\ (tutorial~\CARPref{tut:latex}). Use both
tables and graphs to report numerical results. Use \n{gnuplot}
(tutorial~\CARPref{tut:gnuplot}) or a related utility for graphs.

