% -*- latex -*-
%%%%%%%%%%%%%%%%%%%%%%%%%%%%%%%%%%%%%%%%%%%%%%%%%%%%%%%%%%%%%%%%
%%%%%%%%%%%%%%%%%%%%%%%%%%%%%%%%%%%%%%%%%%%%%%%%%%%%%%%%%%%%%%%%
%%%%
%%%% This text file is part of the source of 
%%%% `The Art of HPC, vol 1: The Science of Computing'
%%%% by Victor Eijkhout, copyright 2012-2023
%%%%
%%%% This book is distributed under a Creative Commons Attribution 3.0
%%%% Unported (CC BY 3.0) license and made possible by funding from
%%%% The Saylor Foundation \url{http://www.saylor.org}.
%%%%
%%%%%%%%%%%%%%%%%%%%%%%%%%%%%%%%%%%%%%%%%%%%%%%%%%%%%%%%%%%%%%%%
%%%%%%%%%%%%%%%%%%%%%%%%%%%%%%%%%%%%%%%%%%%%%%%%%%%%%%%%%%%%%%%%

\usepackage{listings,mdframed,xcolor}
% optional file name, obligatory snippet name
\newcommand\verbatimsnippet   [2][]{\verbatiminput{\codesnippetsdir/#2}}
\newcommand\cverbatimsnippet  [2][]{\lverbatimsnippet{#1}{#2}{C}}
\newcommand\cxxverbatimsnippet[2][]{\lverbatimsnippet{#1}{#2}{C++}}
\newcommand\fverbatimsnippet  [2][]{\lverbatimsnippet{#1}{#2}{Fortran}}
% #1 file name #2 snippet #3 language
\newcommand\lverbatimsnippet  [3]{%
    \IfFileExists
        {\codesnippetsdir/#2}
        {\lstinputlisting[language=#3]{\codesnippetsdir/#2}}
        {\par MISSING SNIPPET #2\par}
}


\def\codedir{code}
\newif\ifMakeOut \MakeOutfalse

\def\getRunOut#1#2{
  \IfFileExists
      {#1/#2.runout}
      {\lstinputlisting[language=verbatim,style=snippetcode,frame=none,xleftmargin=0pt]
        {#1/#2.runout}}
      {\textbf{Missing snippet: #1/#2.runout}}
}

\makeatletter
\def\codefraction{.6}
\def\outputfraction{.35}
\begin{comment}
\newcommand{\answerwithoutput}[3]{
  \moveright .5\unitindent
  \hbox{%
    \begin{minipage}[t]{\codefraction\hsize}
      \def\verbatim@startline{\verbatim@line{\leavevmode\relax}}
      \footnotesize\textbf{Code:}
      \lstset{xleftmargin=0pt}
      \lstinputlisting{\codesnippetsdir/#1}
      \lstset{xleftmargin=.5\unitindent}
    \end{minipage}
    \begin{minipage}[t]{\outputfraction\hsize}
      { \footnotesize \raggedright \textbf{Output\\\relax [#2] #3:}\par }
      \getRunOut{#2}{#3}
    \end{minipage}
  }
}
\end{comment}

\newif\ifMakeOut \MakeOutfalse
\ifInBook
    \def\codesize{\ttfamily\footnotesize}
    \def\verbsize{\ttfamily\footnotesize}
\else
    \def\codesize{\ttfamily\scriptsize}
    \def\verbsize{\ttfamily\scriptsize}
\fi
\newcommand\snippetcodefraction{.57}
\newcommand\snippetanswfraction{.4}
\def\codesnippet#1{%
  \IfFileExists
      {\codesnippetsdir/#1}
      {\lstinputlisting[style=snippetcode,basicstyle=\codesize]{\codesnippetsdir/#1}}
      {\textbf{Missing snippet: #1, not found in \codesnippetsdir}}
}
\usepackage{mdframed,xcolor}

\newcommand{\answerwithoutput}[3]{
  \message{Code snippet <#1> in directory <#2> from program <#3>}
  % go into vertical mode
  \par
  % make nice two-column layout
  \vbox{\leavevmode
    \begin{minipage}[t]{\snippetcodefraction\hsize}
      \begin{mdframed}[backgroundcolor=blue!10]%{quote}
        \def\verbatim@startline{\verbatim@line{\leavevmode\relax}}
        \codesize\textbf{Code:}
        \codesnippet{#1}
      \end{mdframed}
    \end{minipage}
    \begin{minipage}[t]{\snippetanswfraction\hsize}
      \begin{mdframed}[backgroundcolor=yellow!80!white!20]%{quote}
        \codesize
        \ifInBook { \raggedright \textbf{Output\\ \relax [#2] #3:}\par }
        \else     { \raggedright \textbf{Output:}\par }
        \fi
        \IfFileExists
            {\codedir /#2/#3.runout}
            {\lstinputlisting[language=verbatim,style=verbatimcode,frame=none,xleftmargin=0pt]
                {\codedir /#2/#3.runout}}
            {\textbf{missing snippet #2/#3.runout : looking in codedir=\codedir}
              \textbf{missing snippet #2/#3.runout : looking in codedir=\codedir}}
      \end{mdframed}
    \end{minipage}
  }
}
\makeatother

\newcommand{\answerwitherror}[3]{
  % go into vertical mode
  \par
  % make nice two-column layout
  \vbox{
  \begin{multicols}{2}
    \def\verbatim@startline{\verbatim@line{\leavevmode\relax}}
    \footnotesize\textbf{Code:}
    \codesnippet{#1}
    \par\hbox{}\vfill\columnbreak
        \textbf{Output [#2] #3:}
    \hbox{}
    \ifMakeOut
        \immediate\write18{ cd \codedir /#2 && make error_#3.o > #3.runout 2>&1 }
    \fi
    \verbatiminput{\codedir /#2/#3.runout}
    \par\hbox{}\vfill\hbox{}
  \end{multicols}
  }
}

\newcommand\snippetwitherror[3]{
  \answerwitherror{#1}{#2}{#3}
  % record this file as bracketed name
%%  \ifInBook \addchaptersource{{#2/#3}} \fi
}

\newcommand{\snippetwithcomment}[2]{
  % go into vertical mode
  \par
  % make nice two-column layout
  \vbox{
  \begin{multicols}{2}
    \def\verbatim@startline{\verbatim@line{\leavevmode\relax}}
    \footnotesize\textbf{Code:}
    \codesnippet{#1}
    \par\hbox{}\vfill\columnbreak
    { \raggedright\small #2 \par }
    \par\hbox{}\vfill\hbox{}
  \end{multicols}
  }
}

\generalcomment
    {inplaceverbatim}
    {\begingroup
      \def\ProcessCutFile{}
      \def\PrepareCutFile{
        \immediate\write\CommentStream{\noexpand\lstset{language=verbatim}}
        \immediate\write\CommentStream{\noexpand\begin{lstlisting}}
        }
      \def\FinalizeCutFile{\immediate\write\CommentStream
        {\string\end{lstlisting}}}
    }
    {\input{\CommentCutFile}
      \endgroup
    }

\let\answerwithcomment\snippetwithcomment

% #1 : optional file name
% #2 : snippet name
% #3 : directory
% #4 : program to run.
\usepackage{xifthen}
\newcommand\snippetwithoutput[4][]{
  \message{snippet <<#1>> <<#2>> <<#3>> <<#4>>}
  \answerwithoutput{#2}{#3}{#4}
  \ifthenelse{\isempty{#1}}
             {\def\snippetfilename{#4}}
             {\def\snippetfilename{#1}}
  % record this file as bracketed name
  %% \ifInBook
  %% \nobreak
  %% \textsl{For the source of this example, see section~\ref{lst:#3/\snippetfilename}}\par
  %% \addchaptersource{{#3/\snippetfilename}}
  %% \fi
}

\newcommand\csnippetwithoutput[4][]{
  \begingroup
  \lstset{language=C}
  \answerwithoutput{#2}{#3}{#4}
  \lstset{language=C}
  \endgroup
}
\newcommand\cxxsnippetwithoutput[4][]{
  \message{cxxsnippetwithoutput <<#1>> <<#2>> <<#3>> <<#4>>}
  \begingroup
  \lstset{language=C++}
  \answerwithoutput{#2}{#3}{#4}
  \lstset{language=C}
  \endgroup
}
\newcommand{\snippetoutput}[2]{
  \message{In directory <#1> running program <#2>}
  % go into vertical mode
  \par
  % make nice two-column layout
  \begin{minipage}[t]{.3\hsize}{%
      \footnotesize \raggedright \textbf{Output\\\relax [#1] #2:}\par }
      \getRunOut{#1}{#2}
  \end{minipage}
}
\newcommand{\snippetmakeoutput}[2]{
  % go into vertical mode
  \par
  \begin{minipage}[t]{.8\hsize}{%
      \footnotesize \raggedright \textbf{Output\\\relax [#1] #2:}\par }
      \getRunOut{#1}{#2}
  \end{minipage}
}

%%%%%%%%%%%%%%%%
%%%%%%%%%%%%%%%% tutorial snippets
%%%%%%%%%%%%%%%%

\def\tutcodefraction{.6}
\def\tutoutputfraction{.35}

\makeatletter
\newcommand{\tutorialwithoutput}[3]{
  \par
  % make nice two-column layout
  \moveright .5\unitindent
  \hbox{%
    \begin{minipage}[t]{\tutcodefraction\hsize}
      \def\verbatim@startline{\verbatim@line{\leavevmode\relax}}
      \footnotesize\textbf{Code:}
      %\lstset{xleftmargin=0pt}
      \lstinputlisting{\tutsnippetsdir/#1}
      \lstset{xleftmargin=.5\unitindent}
    \end{minipage}
    \begin{minipage}[t]{\tutoutputfraction\hsize}
      { \footnotesize \raggedright \textbf{Output\\\relax [#2] #3:}\par }
      \getRunOut{#2}{#3}
    \end{minipage}
  }
}
\makeatother

%%
%% Makefile displays
%%

\newcommand\makewithoutput[3]{
  \begingroup
  \lstset{language=Bash}
  \def\tutsnippetsdir{#2}
  \def\codefraction{.3}
  \def\outputfraction{.65}
  \tutorialwithoutput{#1}{#2}{#3}
  \lstset{language=C}
  \endgroup
}

\begin{comment}
  \newcommand\makewithoutput[3]{
    \begingroup
    \lstset{language=Bash}
    \def\snippetsdir{#2}
    \answerwithoutput{#1}{#2}{#3}
    \lstset{language=C}
    \endgroup
  }
\end{comment}

%%%%
%%%% input file without header
%%%%
\newcommand\strippedinput[2]{
  \immediate\write18{
    cd #1 && ls #2 
    && awk 'p==1 && !/codesnippet/ {print} NF==0 {p=1}' #2 > #2.stripped.out
  }
  \IfFileExists
      {#1/#2.stripped.out}
      {\lstinputlisting{#1/#2.stripped.out}}
      {\message{Could not strip file #1/#2}
        \lstset{basicstyle=\tiny\ttfamily}
        \lstinputlisting{#1/#2}}
}
