% -*- latex -*-
%%%%%%%%%%%%%%%%%%%%%%%%%%%%%%%%%%%%%%%%%%%%%%%%%%%%%%%%%%%%%%%%
%%%%%%%%%%%%%%%%%%%%%%%%%%%%%%%%%%%%%%%%%%%%%%%%%%%%%%%%%%%%%%%%
%%%%
%%%% This text file is part of the source of 
%%%% `Introduction to High-Performance Scientific Computing'
%%%% by Victor Eijkhout, copyright 2012-2022
%%%%
%%%% This book is distributed under a Creative Commons Attribution 3.0
%%%% Unported (CC BY 3.0) license and made possible by funding from
%%%% The Saylor Foundation \url{http://www.saylor.org}.
%%%%
%%%% bookmacs : macros specifically for the book version
%%%%
%%%%%%%%%%%%%%%%%%%%%%%%%%%%%%%%%%%%%%%%%%%%%%%%%%%%%%%%%%%%%%%%
%%%%%%%%%%%%%%%%%%%%%%%%%%%%%%%%%%%%%%%%%%%%%%%%%%%%%%%%%%%%%%%%

%% skip stuff that's only for the html version
\excludecomment{htmlequation}

%%
%% refer to sections in the HPC book
%% needed for the carpentry book
%%
\usepackage{xr-hyper}
% only in tutorials \externaldocument[HPSC-]{scicompbook}
\newcommand\HPSCref[2][section]{HPC book, #1~\ref{HPSC-#2}}
% only in book \externaldocument[CARP-]{scicomptutorials}
\newcommand\CARPref[2][section]{%
        \textsl{Tutorials book, #1~\ref{CARP-#2}}}
\externaldocument[PCSE-]{parcompbook}
\newcommand\PCSEref[2][section]{%
        \textsl{Parallel Programming book, #1~\ref{PCSE-#2}}}
\externaldocument[ISP-]{ispbook}
\newcommand\ISPref[2][section]{%
        \textsl{Introduction to Scientific Programming book, #1~\ref{ISP-#2}}}
\begin{notlulu}
  \usepackage[xetex,colorlinks,pagebackref=true]{hyperref}
  \hypersetup{bookmarksopen=true}
  \renewcommand*{\backref}[1]{}
  \renewcommand*{\backrefalt}[4]{[{\tiny%
      \ifcase #1 Not cited.%
      \or Cited on page~#2.%
      \else Cited on pages #2.%
      \fi%
  }]}
  \usepackage[all]{hypcap}
\end{notlulu}
\begin{lulu}
  \usepackage{url}
\end{lulu}

%%%%
%%%% layout and graphics
%%%%
\renewcommand\topfraction{.9}
\usepackage{graphicx,wrapfig}
\usepackage{makeidx}

\newcounter{lesson} % for lesson tables
\newcommand\furtherreading{\Level 0 {Further Reading}\label{sec:furtherreading-\chapshortname}}

%%%%%%%%%%%%%%%%%%%%%%%%%%%%%%%%%%%%%%%%%%%%%%%%%%%%%%%%%%%%%%%%
%%%%
%%%% Outlining
%%%%
%%%%%%%%%%%%%%%%%%%%%%%%%%%%%%%%%%%%%%%%%%%%%%%%%%%%%%%%%%%%%%%%
\usepackage{outliner}
\OutlineLevelStart0{\chapter{#1}
  \maillink{arabic{chapter} "#1"}{#1}{chapter}
}
\OutlineLevelStart1{\section{#1}
\maillink{\arabic{chapter}.\arabic{section}}{#1}{section}
}
\OutlineLevelCont1{\section{#1}
\maillink{\arabic{chapter}.\arabic{section}}{#1}{section}
}
\OutlineLevelStart2{\subsection{#1}
  \maillink
    {\arabic{chapter}.\arabic{section}.\arabic{subsection}}{#1}{subsection}
}
\OutlineLevelStart3{\subsubsection{#1}}
\setcounter{secnumdepth}{4}
\OutlineLevelStart4{\paragraph{\bf #1}}

\newcommand\heading[1]{\paragraph*{\textbf{#1}}}

\newenvironment{specialnote}[1]
               {\par\begin{raggedright}
                 \leftskip=.5\unitindent
                 \noindent\textsl{#1.}\kern1em\ignorespaces}
               {\end{raggedright}\par}
\newenvironment{taccnote}
  {\begin{specialnote}{TACC note}}
  {\end{specialnote}}
\newenvironment{enrichment}
  {\begin{specialnote}{Enrichment}}
  {\end{specialnote}}
\newenvironment{zshnote}
  {\begin{specialnote}{Zsh note}}
  {\end{specialnote}}
\newenvironment{fortrannote}
  {\begin{specialnote}{Fortran note}}
  {\end{specialnote}}
\newenvironment{pythonnote}
  {\begin{specialnote}{Python note}}
  {\end{specialnote}}

\newcommand\ListStrippedSource[1]{
  \begingroup \footnotesize
  \immediate \write 18 { ./stripsource #1 }
  \verbatiminput{input.cut}
  \endgroup
}
%%
%% #1 optional language
%% #2 caption
%% #3 full path
%%
\newcommand\ListNamedSource[3][bash]{
  \message{Using language #1}
  \begingroup \footnotesize
  \immediate \write 18 { ./stripsource #3 }
  \lstset{language=#1}
  \srclisting{#2}{input.cut}
  \endgroup
}

%%%%%%%%%%%%%%%%%%%%%%%%%%%%%%%%%%%%%%%%%%%%%%%%%%%%%%%%%%%%%%%%
%%%%
%%%% more environments
%%%%
%%%%%%%%%%%%%%%%%%%%%%%%%%%%%%%%%%%%%%%%%%%%%%%%%%%%%%%%%%%%%%%%

\newenvironment{stufflearned}[1][Commands learned in this section]{%
\begin{tabular}{lp{3in}}
  \toprule
  #1\\
  \midrule
}{%
  \bottomrule
\end{tabular}
}

% this one should probably go?
\newenvironment{fntable}[1]
{\par\hbox\bgroup\kern\unitindent
\begin{tabular}{#1}\toprule}
{\bottomrule\end{tabular}\egroup\par}
