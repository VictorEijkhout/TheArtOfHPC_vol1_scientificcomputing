% -*- latex -*-
%%%%%%%%%%%%%%%%%%%%%%%%%%%%%%%%%%%%%%%%%%%%%%%%%%%%%%%%%%%%%%%%
%%%%%%%%%%%%%%%%%%%%%%%%%%%%%%%%%%%%%%%%%%%%%%%%%%%%%%%%%%%%%%%%
%%%%
%%%% This text file is part of the source of 
%%%% `The Art of HPC, vol 1: The Science of Computing'
%%%% by Victor Eijkhout, copyright 2012-8
%%%%
%%%% This book is distributed under a Creative Commons Attribution 3.0
%%%% Unported (CC BY 3.0) license and made possible by funding from
%%%% The Saylor Foundation \url{http://www.saylor.org}.
%%%%
%%%% chapter on quantum computing
%%%%
%%%%%%%%%%%%%%%%%%%%%%%%%%%%%%%%%%%%%%%%%%%%%%%%%%%%%%%%%%%%%%%%
%%%%%%%%%%%%%%%%%%%%%%%%%%%%%%%%%%%%%%%%%%%%%%%%%%%%%%%%%%%%%%%%

\def\dirac#1{\left|#1\right\rangle}

\Level 0 {Introduction}

We notate `classical bits', or \indextermdef{cbit}s, either as vectors
with two components or with a \emph{Dirac vector
  notation}\index{vector!Dirac notation}:
\[
\dirac{0} \equiv
\begin{pmatrix}
  1\\0
\end{pmatrix},\qquad
\dirac{1} \equiv
\begin{pmatrix}
  0\\1
\end{pmatrix}
\]
Think that location of the~$1$ determines value.

On a single bit we can define four operations, that we represent by
matrices:
\begin{itemize}
\item Identity
  \[ 
  \begin{pmatrix}
    1&0\\ 0&1
  \end{pmatrix}
  \begin{pmatrix}
    1\\0
  \end{pmatrix}
  =
  \begin{pmatrix}
    1\\0
  \end{pmatrix}
  ,\qquad
  \begin{pmatrix}
    1&0\\ 0&1
  \end{pmatrix}
  \begin{pmatrix}
    0\\1
  \end{pmatrix}
  =
  \begin{pmatrix}
    0\\1
  \end{pmatrix}
  \]
\item Negation
  \[ 
  \begin{pmatrix}
    0&1\\ 1&0
  \end{pmatrix}
  \begin{pmatrix}
    1\\0
  \end{pmatrix}
  =
  \begin{pmatrix}
    0\\1
  \end{pmatrix}
  ,\qquad
  \begin{pmatrix}
    0&1\\ 1&0
  \end{pmatrix}
  \begin{pmatrix}
    0\\1
  \end{pmatrix}
  =
  \begin{pmatrix}
    1\\0
  \end{pmatrix}
  \]
\item Constant zero
\item Constant one.
\end{itemize}

Of these operations, the first two are reversible: given the output
you can find the input.

A \indexterm{tensor product} of two vectors has a a result with a
size that is the product of the sizes of the inputs.
\[
\begin{pmatrix}
  x_0\\x_1
\end{pmatrix}
\otimes
\begin{pmatrix}
  y_0\\y_1
\end{pmatrix}
=
\begin{pmatrix}
  x_0
  \begin{pmatrix} y_0\\y_1 \end{pmatrix} \\
  x_1
  \begin{pmatrix} y_0\\y_1 \end{pmatrix} \\
\end{pmatrix}
=
\begin{pmatrix}
  x_0 y_0\\ x_0y_1\\ x_1y_0\\ x_1y_1
\end{pmatrix}
,\qquad
\begin{pmatrix}
  x_0\\x_1
\end{pmatrix}
\otimes
\begin{pmatrix}
  y_0\\y_1
\end{pmatrix}
\otimes
\begin{pmatrix}
  z_0\\z_1
\end{pmatrix}
=
\begin{pmatrix}
  x_0 y_0 z_0\\ x_0 y_0 z_1 \\
  x_0 y_1 z_0\\ x_0 y_1 z_1\\
  x_1 y_0 z_0\\ x_1 y_0 z_1 \\
  x_1 y_1 z_0\\ x_1 y_1 z_1\\
\end{pmatrix}
\]

We use this tensor notation for a multi-bit state in Dirac notation:
\[ \dirac{00} = 
\begin{pmatrix}
  1\\0
\end{pmatrix}
\otimes
\begin{pmatrix}
  1\\0
\end{pmatrix}
=
\begin{pmatrix}
  1\\ 0\\ 0\\ 0
\end{pmatrix}
,\qquad
\dirac{3}=\dirac{11}=
\begin{pmatrix}
  0\\1
\end{pmatrix}
\otimes
\begin{pmatrix}
  0\\1
\end{pmatrix}
=
\begin{pmatrix}
  0\\ 0\\ 0\\ 1
\end{pmatrix}
\]
In general, a string of $n$ bits is represented as a vector of
size~$2^n$.


The \indextermdef{cnot} gate $C$ works on two bits, where the most significant
bit is a `control bit': if it is~1, the other bit, the input bit, is
flipped, otherwise it is reproduced.
\[
C\dirac{00} = 
\begin{pmatrix}
  1&0&0&\\ 0&1&0&0\\ 0&0&0&1\\ 0&0&1&0
\end{pmatrix}
\begin{pmatrix}
  1\\ 0\\ 0\\ 0
\end{pmatrix}
=
\begin{pmatrix}
  1\\ 0\\ 0\\ 0
\end{pmatrix}
=\dirac{00}
,\qquad
C\dirac{10} = 
\begin{pmatrix}
  1&0&0&\\ 0&1&0&0\\ 0&0&0&1\\ 0&0&1&0
\end{pmatrix}
\begin{pmatrix}
  0\\ 0\\ 1\\ 0
\end{pmatrix}
=
\begin{pmatrix}
  0\\ 0\\ 0\\ 1
\end{pmatrix}
=\dirac{11}
\]
The cnot gate is of course reversible.

\Level 1 {Qbits}

We now generalize cbits to \indextermdef{qbit}s: any two-component
vector $(a b)^t$ with $a,b$ complex and
$\|a\|^2+\|b\|^2=1$. Such a qbit is also called a superposition of the
cbits zero and one.

We can measure a qbit, which means map it to~0 with
probabiity~$\|a\|^2$ and to~1 with probability~$\|b\|^2$.
The cbits introduced above collapse to 0 or~1 with chance~1.
As an example, the qbit
$(\sqrt{2}/2\, \sqrt{2}/2)^t$ collapses to 0 or 1 both with chance~$1/2$.

It is easy to prove that the tensor product preserves the unit norm:
\[ 
\begin{pmatrix}
  a\\b
\end{pmatrix}
\otimes
\begin{pmatrix}
  c\\d
\end{pmatrix}
=
\begin{pmatrix}
  ac\\ad\\bc\\bd
\end{pmatrix}
\quad\hbox{and}\quad
\|ac\|^2+\|ad\|^2+\|bc\|^2+\|bd\|^2=1
\]

We can perform all the above matrix operators on qbits.

For instance, the \indexterm{Hadamard gate} takes a zero or one bit
and puts it into equal superposition:
\[ H\dirac{0} =
\begin{pmatrix}
  \sqrt{2}/2&\sqrt{2}/2\\ \sqrt{2}/2&-\sqrt{2}/2
\end{pmatrix}
\begin{pmatrix}
  1\\0
\end{pmatrix}
=
\begin{pmatrix}
  \sqrt{2}/2\\ \sqrt{2}/2
\end{pmatrix}
,\qquad
\begin{pmatrix}
  \sqrt{2}/2&\sqrt{2}/2\\ \sqrt{2}/2&-\sqrt{2}/2
\end{pmatrix}
\begin{pmatrix}
  0\\1
\end{pmatrix}
=
\begin{pmatrix}
  \sqrt{2}/2\\ -\sqrt{2}/2
\end{pmatrix}
\]
This operation is reversible, so it can take an equal superposition
and turn it into a zero or one bit.

\Level 1 {Deutsch oracle}

Suppose you have a function on one bit. (See above: there are four of
those.) You need two queries (test inputs) to determine what the
function is. On the other hand, you could ask if the function is
constant (identically zero or one), or variable (that is, either
identity or negation). Classically this again takes two queries, but
on a quantum computer a single query suffices.

