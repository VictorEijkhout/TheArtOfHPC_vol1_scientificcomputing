% -*- latex -*-
%%%%%%%%%%%%%%%%%%%%%%%%%%%%%%%%%%%%%%%%%%%%%%%%%%%%%%%%%%%%%%%%
%%%%%%%%%%%%%%%%%%%%%%%%%%%%%%%%%%%%%%%%%%%%%%%%%%%%%%%%%%%%%%%%
%%%%
%%%% This text file is part of the source of 
%%%% `Introduction to High-Performance Scientific Computing'
%%%% by Victor Eijkhout, copyright 2012-2022
%%%%
%%%% This book is distributed under a Creative Commons Attribution 3.0
%%%% Unported (CC BY 3.0) license and made possible by funding from
%%%% The Saylor Foundation \url{http://www.saylor.org}.
%%%%
%%%%
%%%%%%%%%%%%%%%%%%%%%%%%%%%%%%%%%%%%%%%%%%%%%%%%%%%%%%%%%%%%%%%%
%%%%%%%%%%%%%%%%%%%%%%%%%%%%%%%%%%%%%%%%%%%%%%%%%%%%%%%%%%%%%%%%

\def\R{{\cal R}}

\Level 0 {Clustering}
\index{clustering|(textbf}

An important part of many data analytics algorithms
is a \emph{clustering} stage, which sorts the data
in a small number of groups that, presumably, belong together.

\Level 1 {$k$-means clustering}

The \indextermsub{$k$-means}{clustering} algorithm sorts data
into a predefined number of clusters. Each cluster is defined by
its center, and a data point belongs to a cluster if it's closer
to the center of that cluster than to that of any other.

\begin{enumerate}
\item Find $k$ distinct points to serve as initial cluster centers.
\item Now repeat until the clusters are stable:
\item Group to the points according to the center they are closets to;
\item recompute the centers based on these points.
\end{enumerate}

\Level 1 {Spectral clustering}
\label{sec:spectral-cluster}

The disadvantage of $k$-means clustering is that it presumes convex clusters.
Other algorithms are more general. For instance, in spectral graph theory
(section~\ref{app:fiedler}) the \indexterm{Fiedler vector} sorts the nodes of a graph
into a small number of connected groups. We could apply this method as follows:
\begin{enumerate}
\item Let $D$ be the distance matrix $D_{ij}=|x_i-x_j|$ of the data points.
\item Let $W$ be the diagonal matrix containing the row sums of~$D$; then
\item $W-D$ is a matrix for which we can find the Fiedler vector.
\end{enumerate}

Normally, we would use \indexterm{inverse iteration} to find this vector,
but it is also possible~\cite{LinCohen:PIC} to use the power method;
section~\ref{app:power-method}.

\index{clustering|)}

\Level 0 {Classification}

\Level 1 {Decision trees}

A \indexterm{decision tree} is a classifier for a combination of ordered and
unordered attributes. For each attribute, we have a list of
\begin{quote}
  record-id $|$ class $|$ value
\end{quote}
triplets.

Building the decision tree now goes as follows:\\
\begin{tabbing}
  until \=all attributes are covered:\\
  \>find an attribute\\
  \>find a way to split the attribute\\
  \>for \=all other attributes\\
  \>\>split\= the attribute lists in\\
  \>\>\>a `left' and `right' part
\end{tabbing}

Finding an attribute is typically done using the \indexterm{gini
  index}. After that, finding a split point in that attribute is
relatively simple for numerical attributes, though this is facilitated
by always \indexterm{sorting} the lists.

Splitting non-numeric, or `categorical', attributes is harder. We make
a count of the records with the various values. We then group these to
give a balanced split.

Spliting the attribute lists takes one
scan per attribute list:\\
\begin{tabbing}
  for \=each item:\\
  \>use\= the record identifier to\\
  \>\>find its value of the split attribute\\
  \>based on this assign the item to\\
  \>\>the `left' or `right' part of this attribute list.
\end{tabbing}

If the splitting attribute is categorical, assigning a record is
slightly harder. For this, in the splitting stage
typically a \indexterm{hash table} is used, noting for each record
number in which subtree it is classified. For the other attribute
lists the record number is then used to move the item left or right.

For further reading, see~\cite{Sprint:classifier,ScalParC}.

\Level 0 {Recommender systems}

The article 
\url{http://www2.research.att.com/~volinsky/papers/ieeecomputer.pdf}
describes systems such as from \indexterm{Netflix}
that try to predict how much a given person will like a given item.

The approach here is identify a, relatively low dimensional, space~$\R^f$
of \indexterm{features}. For movies one could for instance adopt two features:
low-brow versus high-brow, and historic vs contemporary.
(High-brow historic: Hamlet; low-brow historic: Robin Hood, men in tights; 
high-brow contemporary: anything by Godard; low-brow contemporary, eh, 
do I really need to find examples?)
For each item (movie)~$i$ there is then a vector $q_i\in\R^f$ that characterises
the item, and for each person there is a vector $p_j\in\R^f$ that
characterises how much that person likes such movies. The predicted rating
for a user of a movie is then $r_{ij}=q_i^tp_j$.

The vectors $q_i,p_j$ are gradually learned based on actual
ratings~$r_{ij}$, for instance by gradually minimizing
\[ \min_{p,q} \sum_{ij} (r_{ij}-q_i^tp_j)^2+
    \lambda \bigl( \|q_i\|^2+\|p_j\|^2 \bigr)
\]

\Level 1 {Stochastic gradient descent}
\index{stochastic gradient descent}

Let $e_{ij}=r_{ij}-q_i^tp_j$ be the difference between actual and predicted rating.
\[
\begin{array}{l}
  q_i\leftarrow q_i+\gamma (e_{ij}p_j-\lambda q_i)\\
  p_j\leftarrow p_j+\gamma (e_{ij}q_i-\lambda p_j)
\end{array}
\]

Decentralized recommending:~\cite{Zheng:2016arXiv:decentral-recommend}

\Level 1 {Alternating least squares}

First fix all $p_j$ and minimize the $q_i$ (in parallel),
then fix $q_i$ and minimize the~$p_j$.
This handles missing data better.

%Machine learning, for an introduction to ML techniques; see~\cite{Tara:MLbiology}.

\Level 0 {Distributed hash tables}

See \indexterm{memcached}
\url{https://en.wikipedia.org/wiki/Memcached}
