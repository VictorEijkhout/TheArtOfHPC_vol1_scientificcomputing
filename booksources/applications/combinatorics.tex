% -*- latex -*-
%%%%%%%%%%%%%%%%%%%%%%%%%%%%%%%%%%%%%%%%%%%%%%%%%%%%%%%%%%%%%%%%
%%%%%%%%%%%%%%%%%%%%%%%%%%%%%%%%%%%%%%%%%%%%%%%%%%%%%%%%%%%%%%%%
%%%%
%%%% This text file is part of the source of 
%%%% `The Art of HPC, vol 1: The Science of Computing'
%%%% by Victor Eijkhout, copyright 2012-6
%%%%
%%%% This book is distributed under a Creative Commons Attribution 3.0
%%%% Unported (CC BY 3.0) license and made possible by funding from
%%%% The Saylor Foundation \url{http://www.saylor.org}.
%%%%
%%%%
%%%%%%%%%%%%%%%%%%%%%%%%%%%%%%%%%%%%%%%%%%%%%%%%%%%%%%%%%%%%%%%%
%%%%%%%%%%%%%%%%%%%%%%%%%%%%%%%%%%%%%%%%%%%%%%%%%%%%%%%%%%%%%%%%


In this chapter we will briefly consider a few combinatorial algorithms:
sorting, and prime number finding with the Sieve of Eratosthenes.

\SetBaseLevel 1
\input applications/sorting
\SetBaseLevel 0

%\begin{notready}
\Level 1 {Prime number finding}

The \indextermsub{sieve of}{Eratosthenes} is a very old method for
\emph{finding prime numbers}\index{prime number!finding}.
In a way, it is still the basis of many more modern methods.

Write down all natural numbers from~2 to some upper bound~$N$,
and we are going to mark these numbers as prime or definitely not-prime.
All numbers are initially unmarked.
\begin{itemize}
\item The first unmarked number is~2: mark it as prime, and mark all its
  multiples as not-prime.
\item The first unmarked number is~3: mark it as prime, and mark all its
  multiples as not-prime.
\item The next number is~4, but it has been marked already, so mark~5
  and its multiples.
\item The next number is~6, but it has been marked already, so mark~7
  and its multiples.
\item Numbers 8,9,10 have been marked, so continue with~11.
\item Et cetera.
\end{itemize}

%\end{notready}
